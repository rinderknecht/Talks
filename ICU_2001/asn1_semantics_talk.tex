%%-*-latex-*-

\documentclass[12pt]{article}

\usepackage{amssymb}
\usepackage{amsmath}
\usepackage{latexsym}
\usepackage{stmaryrd}
\usepackage{xspace}
\usepackage{url}
\usepackage{zzslides}

%%-*-latex-*-

%% ASN.1 Acronyms
%%
\newcommand{\ASN}{\mbox{ASN.1}\xspace}

\newcommand{\OCaml}{\textrm{OCaml}\xspace}


%% Abstract Syntax Trees
%%
\newcommand{\ocamlkwd}[1]{\textnormal{\textsf{\textbf{#1}}}}
\newcommand{\ocamltypename}[1]{\textnormal{\textsl{#1}}}
\newcommand{\ocamlvaluename}[1]{\textnormal{\textit{#1}}}
\newcommand{\asnkwdconstr}[1]{\textsc{#1}}
\newcommand{\asnkwdpvar}[1]{\textsf{`}\textsc{#1}}
\newcommand{\ocamlconstr}[1]{\textnormal{\textsf{#1}}}
\newcommand{\ocamlpvar}[1]{\textnormal{\textsf{`#1}}}
%%\newcommand{\asnsym}[1]{\textsf{\textbf{#1}}\xspace}
\newcommand{\asnsymbol}[1]{\boldsymbol{#1}}
%\newcommand{\wildcard}{\textnormal{\Huge \_\,}}
\newcommand{\wildcard}{\textnormal{\large \raisebox{0.5mm}{\_}}\xspace}
\newcommand{\wildcardof}{\wildcard \ocamlkwd{of}\xspace}
\newcommand{\ocamlmodulepath}[1]{\textnormal{#1}}
\newcommand{\ocamlcomment}[1]{\begin{flushleft} \textit{(* #1 *)} \end{flushleft}}
%%\newcommand{\quotedasnsymbol}[1]{\textsf{"}$\asnsymbol{#1}$\,\textsf{"}}
\newcommand{\backquote}{\textbf{\textsf{`}}}

\newcommand{\LEQ}{\pmb{\leqslant}}
\newcommand{\LT}{\pmb{<}}

\newcommand{\Some}{\ocamlconstr{Some}}
\newcommand{\None}{\ocamlconstr{None}}

\newcommand{\where}{\ocamlkwd{where}}

\newcommand{\kwdCHOICE}{\asnkwdconstr{Choice}}
\newcommand{\kwdSET}{\asnkwdconstr{Set}}
\newcommand{\kwdSEQUENCE}{\asnkwdconstr{Sequence}}
\newcommand{\kwdSETOF}{\asnkwdconstr{Set of}}
\newcommand{\kwdSEQUENCEOF}{\asnkwdconstr{Sequence of}}
\newcommand{\kwdINTEGER}{\asnkwdconstr{Integer}}
\newcommand{\kwdBITSTRING}{\asnkwdconstr{Bit string}}
\newcommand{\kwdENUMERATED}{\asnkwdconstr{Enumerated}}
\newcommand{\kwdREAL}{\asnkwdconstr{Real}}
\newcommand{\kwdBOOLEAN}{\asnkwdconstr{Boolean}}
\newcommand{\kwdNULL}{\asnkwdconstr{Null}}
\newcommand{\kwdOCTETSTRING}{\asnkwdconstr{Octet string}}
\newcommand{\kwdOBJECTIDENTIFIER}{\asnkwdconstr{Object identifier}}
\newcommand{\kwdRELATIVEOID}{\asnkwdconstr{Relative OID}}
\newcommand{\kwdEMBEDDEDPDV}{\asnkwdconstr{Embedded PDV}}
\newcommand{\kwdEXTERNAL}{\asnkwdconstr{External}}
\newcommand{\kwdCHARACTERSTRING}{\asnkwdconstr{Character string}}
\newcommand{\kwdOPTIONAL}{\asnkwdconstr{Optional}}
\newcommand{\kwdDEFAULT}{\asnkwdconstr{Default}}
\newcommand{\kwdCOMPONENTSOF}{\asnkwdconstr{Components of}}
\newcommand{\TRef}{\ocamlconstr{TRef}}
\newcommand{\VRef}{\ocamlpvar{VRef}}
\newcommand{\kwdTRUE}{\asnkwdconstr{True}}
\newcommand{\kwdFALSE}{\asnkwdconstr{False}}
\newcommand{\Null}{\ocamlpvar{Null}}
\newcommand{\kwdMINUSINFINITY}{\asnkwdconstr{Minus-infinity}}
\newcommand{\kwdPLUSINFINITY}{\asnkwdconstr{Plus-infinity}}
\newcommand{\kwdMIN}{\asnkwdconstr{Min}}
\newcommand{\kwdMAX}{\asnkwdconstr{Max}}
\newcommand{\MinInfInt}{\ocamlpvar{MinInfInt}}
\newcommand{\PlusInfInt}{\ocamlpvar{PlusInfInt}}
\newcommand{\MinInfReal}{\ocamlpvar{MinInfReal}}
\newcommand{\PlusInfReal}{\ocamlpvar{PlusInfReal}}
\newcommand{\BinStr}{\ocamlpvar{BinStr}}
\newcommand{\PosInt}{\ocamlpvar{PosInt}}
\newcommand{\NegInt}{\ocamlpvar{NegInt}}
\newcommand{\Real}{\ocamlconstr{Real}}
\newcommand{\PosReal}{\ocamlpvar{PosReal}}
\newcommand{\NegReal}{\ocamlpvar{NegReal}}
\newcommand{\String}{\ocamlconstr{String}}
\newcommand{\pvString}{\ocamlpvar{String}}
\newcommand{\RestrStr}{\ocamlconstr{RestrStr}}
\newcommand{\Regexp}{\ocamlpvar{Regexp}}
\newcommand{\Ident}{\ocamlpvar{Ident}}
\newcommand{\Enum}{\ocamlpvar{Enum}}
\newcommand{\List}{\ocamlpvar{List}}
\newcommand{\Map}{\ocamlpvar{Map}}
\newcommand{\Nil}{\ocamlpvar{Nil}}
\newcommand{\Cons}{\ocamlpvar{Cons}}
\newcommand{\Bind}{\ocamlpvar{Bind}}
\newcommand{\Comp}{\ocamlconstr{Comp}}
\newcommand{\kwdAPPLICATION}{\asnkwdconstr{Application}}
\newcommand{\kwdUNIVERSAL}{\asnkwdconstr{Universal}}
\newcommand{\kwdPRIVATE}{\asnkwdconstr{Private}}
\newcommand{\kwdEXPLICIT}{\asnkwdconstr{Explicit}}
\newcommand{\kwdIMPLICIT}{\asnkwdconstr{Implicit}}
\newcommand{\kwdAUTOMATIC}{\asnkwdconstr{Automatic}}
\newcommand{\kwdTAGS}{\asnkwdconstr{Tags}}
\newcommand{\kwdPRESENT}{\asnkwdconstr{Present}}
\newcommand{\kwdABSENT}{\asnkwdconstr{Absent}}
\newcommand{\kwdUNION}{\asnkwdconstr{Union}}
\newcommand{\kwdINTERSECTION}{\asnkwdconstr{Intersection}}
\newcommand{\kwdEXCEPT}{\asnkwdconstr{Except}}
\newcommand{\kwdALLEXCEPT}{\asnkwdconstr{All except}}
\newcommand{\kwdFROM}{\asnkwdconstr{From}}
\newcommand{\kwdSIZE}{\asnkwdconstr{Size}}
\newcommand{\kwdINCLUDES}{\asnkwdconstr{Includes}}
\newcommand{\kwdWITHCOMPONENT}{\asnkwdconstr{With component}}
\newcommand{\kwdWITHCOMPONENTS}{\asnkwdconstr{With components}}
\newcommand{\kwdPATTERN}{\asnkwdconstr{Pattern}}
\newcommand{\Value}{\ocamlconstr{Value}}
\newcommand{\HexStr}{\ocamlpvar{HexStr}}
\newcommand{\ObjId}{\ocamlconstr{ObjId}}
\newcommand{\Empty}{\ocamlpvar{Empty}}
\newcommand{\Universal}{\ocamlconstr{Universal}}
\newcommand{\Partial}{\ocamlconstr{Partial}}
\newcommand{\Full}{\ocamlconstr{Full}}
\newcommand{\Interval}{\ocamlpvar{Interval}}

\newcommand{\T}{\textnormal{\textrm{T}}}
\newcommand{\normT}{\overline{\T}}
\newcommand{\C}{\textnormal{\textrm{C}}}
%%\newcommand{\F}{{\cal F}}
\newcommand{\I}{{\cal I}}
\newcommand{\J}{{\cal J}}
\newcommand{\V}{{\cal V}}
\newcommand{\normV}{\overline{\cal V}}

\newcommand{\Labels}{{\cal L}}

\newcommand{\Env}[2]{{#1}\hspace*{-0.5mm}_{_{#2}}}
\newcommand{\FieldEnv}[1]{\Env{\textnormal{\textrm{F}}}{#1}}
\newcommand{\CompEnv}[1]{\Env{\Phi}{#1}}
\newcommand{\IdEnv}[1]{\Env{\textnormal{\textrm{M}}}{#1}}
\newcommand{\IntEnv}[1]{\Env{\textnormal{\textrm{L}}}{#1}}
\newcommand{\FieldConst}[1]{\Env{\textnormal{\textrm{K}}}{#1}}
\newcommand{\NamedBits}[1]{\Env{{\cal N}}{#1}}

\newcommand{\NormFieldConst}[1]{\textnormal{\textrm{K}}'\hspace*{-1.3mm}_{_{#1}}}
\newcommand{\NormCompEnv}[1]{\Phi'\hspace*{-1.3mm}_{_{#1}}}

\newcommand{\Herbrand}{{\cal H}}
\newcommand{\LIST}{\textnormal{L}}
\newcommand{\MAP}{{\cal Q}}
\newcommand{\NEWTERMS}{\textnormal{N}}
\newcommand{\ASSIGN}{\textnormal{\textrm{::=}\xspace}}

\newcommand{\TSE}{\denotC}
\newcommand{\R}{{\cal R}}
\newcommand{\TSEmap}[1]{\MAP_{_{#1}}}
\newcommand{\TSEnormMap}[1]{\MAP'_{#1}}

\newcommand{\IntInter}{\ocamlconstr{IntInter}}
\newcommand{\RealInter}{\ocamlconstr{RealInter}}
\newcommand{\StrInter}{\ocamlconstr{StrInter}}

\newcommand{\SC}{\textnormal{SC}}
\newcommand{\VAR}{\textnormal{V}}

\newcommand{\MU}[2]{\mu_{#1} \, (#2)}


%%-*-latex-*-

%% New commands for typesetting Inference Rules
%%

\newcommand{\Path}{{\cal P}} %%{\textrm{P}}
%%\newcommand{\Pt}{\textrm{P}_{\textrm{t}}}
%%\newcommand{\Ph}{\textrm{P}_{\textrm{h}}}

\newcommand{\E}{\textnormal{E}}

\newcommand{\denotC}[4]{\llbracket #2 \rrbracket_{#3}(#4)}

\newcommand{\card}[1]{\mid\!#1\!\mid}

\newcommand{\domain}[1]{\langle #1 \rangle}

\newcommand{\TypeEnv}{\Gamma}
\newcommand{\ValueEnv}{\Delta}

\newcommand{\wild}{\mbox{\large \raisebox{0.5mm}{\_}}\xspace}
\renewcommand{\colon}{\textsf{\textbf{\,:\,}}}

%% Boolean Operators
%%
\newcommand{\BOOLand}{\mbox{\scriptsize \, $\wedge$ \,}}
\newcommand{\BOOLor}{\mbox{\scriptsize \, $\vee$ \,}}

%% Operators on Set Expressions (SE)
%%
\newcommand{\SEmu}{\dot{\mu}}
\newcommand{\SEcap}{\dot{\cap}}
\newcommand{\SEbigcap}{\dot{\bigcap}}
\newcommand{\SEcup}{\dot{\cup}}
\newcommand{\SEbigcup}{\dot{\bigcup}}
\newcommand{\SEdiff}{\dot{\backslash}}
\newcommand{\SEneg}{\dot{\neg}}
\newcommand{\SEdiam}{\dot{\Diamond}}
\newcommand{\SEbot}{\dot{\textnormal{\textsf{0}}}}
\newcommand{\SEtop}{\dot{\textnormal{\textsf{1}}}}

\newcommand{\eval}[2]{[#1,#2]}

%% Operators on Powerset Expressions (PSE)
%%
\newcommand{\PSEsum}{\oplus}
\newcommand{\PSEprod}{\otimes}
\newcommand{\PSEdiff}{\ominus}

%% Operators on Constraints Expressions (CE)
%%
\newcommand{\CEmu}{\mbox{\raisebox{-0.5mm}{$\stackrel{\pmb{..}}{\mu}$}}}
\newcommand{\CEand}{\mbox{\scriptsize \, $\stackrel{\pmb{..}}{\wedge}$ \,}}
\newcommand{\CEor}{\mbox{\scriptsize \, $\stackrel{\pmb{..}}{\vee}$ \,}}
\newcommand{\CEbigand}{\stackrel{\pmb{..}}{\bigwedge}}
\newcommand{\CEsubseteq}{\stackrel{\pmb{..}}{\subseteq}}
\newcommand{\ddsubseteq}[1]{\CEsubseteq_{\mbox{\raisebox{-0.6mm} {$\scriptscriptstyle \textnormal{#1}$}}}}
\newcommand{\CEsubseteqI}{\ddsubseteq{I}}
\newcommand{\CEsubseteqR}{\ddsubseteq{R}}
\newcommand{\CEsubseteqS}{\ddsubseteq{S}}
\newcommand{\CEeq}{\stackrel{\pmb{..}}{=}}
\newcommand{\ddeq}[1]{\CEeq_{\mbox{\raisebox{-0.6mm} {$\scriptscriptstyle \textnormal{#1}$}}}}
\newcommand{\CEeqI}{\ddeq{I}}
\newcommand{\CEeqR}{\ddeq{R}}
\newcommand{\CEeqS}{\ddeq{S}}
\newcommand{\CEeqF}{\ddeq{F}}

%% Operators on Powerset Constraint Expressions (PSCE)
%%
\newcommand{\PSCEeq}{\ddeq{P}}


%% Operators on Patterns
%%
\newcommand{\PAarrow}{\,\, {\boldsymbol\rightarrow}}



\newcommand{\disjunion}{%
\setbox0=\hbox{$\sqcup$}%
\setbox1=\hbox{\toto{\boldmath{+}}}%
\dimen0=\the\wd0%
\dimen1=\the\wd1%
\advance\dimen0 by\the\dimen1%
\divide\dimen0 by2%
\mathbin{\vbox{\hphantom{$\sqcup$}\nointerlineskip\hbox{\box0\kern -\dimen0\raise 0.6mm\box1}}}}

%% Operators on Caml Lists
%%
\newcommand{\CONS}{\!::\!}
\newcommand{\Append}{\, \mbox{\sf \small @} \,}

%% Other Stuff
%%

%\SList{T}{i}{n} => {T}$_{1 <= i <= n}$
\newcommand{\SList}[3]{\mbox{\{#1\}$_{\scriptscriptstyle{1} \scriptscriptstyle{\leqslant} #2 \scriptscriptstyle{\leqslant} #3}$}}

%\OList{T}{i}{n} => [T]$_{1 <= i <= n}$
\newcommand{\OList}[3]{\mbox{[#1]$_{\scriptscriptstyle{1} \scriptscriptstyle{\leqslant} #2 \scriptscriptstyle{\leqslant} #3}$}}

\newcommand{\REAL}{\textrm{"REAL"}}
\newcommand{\INTEGER}{\textrm{"INTEGER"}}

\newcommand{\emptyL}{\mbox{\rm [\,]\,}}
\newcommand{\emptyStr}{""}

%%\newcommand{\Field}[1]{\mbox{{\sf Field} ($l_#1$, $\tau_#1$, T$_#1$, $\sigma_#1$, $s_#1$)}}

\newcommand{\CHOICE}[2]{\kwdCHOICE \, \OList{($l_#1$, ($\tau_#1$, T$_#1$, $\sigma_#1$)}{#1}{#2}}

\newcommand{\SET}[2]{\kwdSET \, \OList{\Field{#1}}{#1}{#2}}
\newcommand{\SEQUENCE}[2]{\kwdSEQUENCE \, \OList{\Field{#1}}{#1}{#2}}

\newcommand{\wildCHOICE}{\kwdCHOICE \,\!\wild}
\newcommand{\wildSET}{\kwdSET \,\!\wild}
\newcommand{\wildSEQUENCE}{\kwdSEQUENCE \,\!\wild}
\newcommand{\wildINTEGER}{\kwdINTEGER \,\!\wild}
\newcommand{\wildENUMERATED}{\kwdENUMERATED \,\!\wild}
\newcommand{\wildTRef}{\asnkwdconstr{TRef}\,\!\wild}
\newcommand{\wildVRef}{\asnkwdconstr{VRef}\,\!\wild}
\newcommand{\wildSelect}{\wild \!<\! \wild}
\newcommand{\wildBITSTRING}{\kwdBITSTRING \, \wild}
\newcommand{\wildSETOF}{\kwdSETOF \, \wild}
\newcommand{\wildSEQUENCEOF}{\kwdSEQUENCEOF \, \wild}
\newcommand{\emptyINTEGER}{\kwdINTEGER $\emptyL$}
\newcommand{\emptyENUMERATED}{\kwdENUMERATED $\emptyL$}
\newcommand{\emptyBITSTRING}{\kwdBITSTRING $\emptyL$}
\newcommand{\emptySET}{\kwdSET $\emptyL$}
\newcommand{\emptySEQUENCE}{\kwdSEQUENCE $\emptyL$}

\newcommand{\AS}{\, \mbox{\bf as} \,}

\newcommand{\vdashL}{\vdash\hspace*{-2.5mm}_{\scriptscriptstyle {\cal L}} \,}
\newcommand{\vdashT}{\vdash\hspace*{-2.5mm}_{\scriptscriptstyle {\cal T}} \,}
\newcommand{\vdashC}{\vdash\hspace*{-2.5mm}_{\scriptscriptstyle {\cal C}} \,}
\newcommand{\vdashD}{\vdash\hspace*{-2.5mm}_{\scriptscriptstyle {\cal D}} \,}
\newcommand{\vdashCOMP}{\vdash\hspace*{-2.5mm}_{\scriptscriptstyle {\rm comp}} \,}
\newcommand{\vdashAUTO}{\vdash\hspace*{-2.5mm}_{\scriptscriptstyle {\rm auto}} \,}
\newcommand{\bouchon}[1]{\vdash\hspace*{-2.5mm}_{\scriptscriptstyle #1} \,\,}
\newcommand{\Bouchon}[1]{\Vdash\hspace*{-2.5mm}_{\scriptscriptstyle #1} \,\,}

\newcommand{\listminus}{-}
\newcommand{\listin}{\in}
\newcommand{\listinter}{\sqcap}
\newcommand{\listunion}{\sqcup}

\newcommand{\permutation}[2]{#1 \,\, \mbox{\em est une permutation sur} \,\, #2}
\newcommand{\perm}[1]{#1 \,\, \mbox{\em est une permutation}}
\newcommand{\noother}{{\bf when} \,\, \mbox{\em no other rule match}}

\newcommand{\poe}{\preccurlyeq}
\newcommand{\bob}{\mbox{\boldmath $\{$}}
\newcommand{\bcb}{\mbox{\boldmath $\}$\,}}


\newcommand{\Wide}{\, \sqsubseteq \,}
\newcommand{\PAR}{\,\,\|\,\,}

\newcommand{\nlhd}{\not{\hspace*{-1mm}\lhd}\hspace*{1mm}}

\newcommand{\clearemptydoublepage}{\newpage{\pagestyle{empty}\cleardoublepage}}


%%\font\toto=cmr5 scaled 700

\newcommand{\rulename}[1]{$\langle\textnormal{\textsf{#1}}\rangle$}

%%\newcommand{\ASN}{\mbox{ASN.1}\xspace}

\landscape
%%\numbersection

\begin{document}

\zztitle{A Set-based Semantics for Validating \ASN Specifications}

\zzauthor{Christian Rinderknecht}

\organization{\large Network Architecture Laboratory\\
Information and Communications University\\
58-4 Hwaam-dong, Yuseong-gu, Daejeon\\
305-732, South Korea\\
\texttt{rinderkn@icu.ac.kr}}
\date{March 28, 2002}

\reminder{Network Architecture Laboratory Seminar}

\titlepage

\section{Plan}
\begin{slide}
\btitle{\textsf{Plan}}

\begin{itemize}

  \item Introduction
 
  \item Short Presentation of \ASN

  \item Issues in Validation

  \item A Constraint-based Solution

  \item Conclusion

\end{itemize}

\bigskip \bigskip \bigskip \bigskip \bigskip \bigskip

\end{slide}

\section{Introduction}
\begin{slide}
\btitle{Introduction}

Abstract Syntax Notation One (\ASN) is a standard language for
defining data types whose values may be exchanged across a network
between two communicating applications, independently from the
possible heterogeneity of the peers.\\

It is used in network management, secure email, mobile
telephony, air traffic control, voice and video over
the Internet, electronic commerce, digital certificates, secure email,
radio paging, interactive television, financial service systems,
biometrics, ATM transactions, 800-number call routing to local
carriers (USA), plane take-offs and landings, Federal Express tracking
network, Microsoft Internet Explorer, NetMeeting, Outlook, wireless
applications from Nokia, Ericsson and Motorola, databases (genetics,
libraries), diagnostic monitoring systems (car factories), subway
equipments, etc.\\

The Encoding Rules specify how to serialize an \ASN value.

\end{slide}


\section{\ASN and the Compilation Chain}
\begin{slide}

\epsfxsize=17cm
\epsfbox{compilation}

\end{slide}

\section{Short Presentation of \ASN}
\begin{slide}
\btitle{\textsf{Short Presentation of \ASN}}

\noindent
\textbf{Basic types}

\begin{itemize}

  \item \verb+ok BOOLEAN ::= TRUE+

%%  \item The \texttt{NULL} type has only one value, also noted
%%        \texttt{NULL};

  \item 

\begin{verbatim} 
zero INTEGER ::= 0 
DayInTheYear ::= INTEGER {first(1), last(356)}
newYearsEve DayInTheYear ::= last
\end{verbatim}

  \item 
\begin{verbatim}
SynchroIndicator ::= ENUMERATED {serial, parallel}
synchro SynchroIndicator ::= serial
\end{verbatim}

  \item

\begin{verbatim}
pi REAL ::= 3.14159
micron REAL ::= {mantissa 1, base 10, exponent -6}
\end{verbatim}

  \item 

\begin{verbatim}
byte BIT STRING ::= 'OD'H  -- or '00001110'B
T ::= BIT STRING {msb(7), lsb(0)}
v T ::= {msb, lsb}  -- or '10000001'B
\end{verbatim}

%%  \item The \texttt{OCTET STRING} type is similar to the \texttt{BIT
%%        STRING} for eight-bit multiples strings.

%%  \item The \texttt{OBJECT IDENTIFIER} and \texttt{RELATIVE-OID} types
%%        aim at referencing other \ASN modules (a set of type
%%        and value declarations) at an international level, by means of
%%        a path in a standard tree.

  \item For historical reasons, there are many string types.

\end{itemize}

%%\continued
\bigskip

\noindent
\textbf{Constructed types}

\begin{itemize}

  \item The \texttt{SET} type corresponds to the record-like
        structures in programming languages:

\begin{verbatim}
PersonInfo ::= SET {age INTEGER,married BOOLEAN}
i PersonInfo ::= {married FALSE, age 32}
\end{verbatim}

        but also some fields can be marked as optional or having a
        default value:

\begin{verbatim}
Point ::= SET {x REAL DEFAULT 0, y REAL DEFAULT 0}
origin Point ::= {}           -- or {x 0.0, y 0.0}
\end{verbatim}

%%\bigskip\bigskip\bigskip\bigskip\bigskip\bigskip

%%\continued

        A real example:

\begin{verbatim}
DataAcknowledgementTPDU ::= SET {
  destRef        Reference,
  yr-tu-nr       TPDUnumber,
  checkSum       CheckSum OPTIONAL,
  subSeqNr       SubSequenceNumber DEFAULT 0,
  flowControlCnf FlowControlConfirmation OPTIONAL}
\end{verbatim}


%%  \item The \texttt{SEQUENCE} type is the same as the \texttt{SET}
%%        type, except that the values for the fields must be given in
%%        the same order as they are declared:
%%\begin{verbatim}
%%PersonInfo ::= SEQUENCE {age INTEGER,married BOOLEAN}
%%i PersonInfo ::= {married FALSE, age 32}    -- WRONG!
%%\end{verbatim}

%%\continued

  \item The \texttt{SET OF} type corresponds to the mathematical
        notion of sets with repetition:

\begin{verbatim}
T ::= SET OF INTEGER
empty T ::= {}
small T ::= {7, 9, 1, 1, 3}
\end{verbatim}

%%  \item The \texttt{SEQUENCE OF} type corresponds to the dynamic
%%        arrays or lists of some programming languages. It is similar
%%        to the \texttt{SET OF} type, except that the elements will be
%%        encoded in the specified order.

%%\continued
\pagebreak

  \item The \texttt{CHOICE} type corresponds to a \textsf{union} in
        C or a \textsf{case} in Pascal.

\begin{verbatim}
T ::= CHOICE {x REAL, y BOOLEAN}
u T ::= x : 0.5
v T ::= y : FALSE
\end{verbatim}

%%\continued

        The Protocol Data Units (PDU) are \texttt{CHOICE} types,
        because they model all the possible queries and responses
        between two peers. A \texttt{CHOICE} type may be recursive,
        like the other constructed types. One real example (a Network
        Management Protocol) is:

\begin{verbatim}
CMISFilter ::= CHOICE {
  item  FilterItem,
  and   SET OF CMISFilter,
  or    SET OF CMISFilter,
  not   CMISFilter}
\end{verbatim}

\end{itemize}

%%\continued 
\pagebreak

\textbf{Subtyping constraints}\\

\ASN offers a very involved subtyping paradigm consisting of
constraints upon recursive types, that restricts their corresponding
sets of values in a set-theoretic manner, but also in a structural
way.

\begin{itemize}

  \item \textsf{Interval Constraint}: the \texttt{INTEGER},
        \texttt{REAL} and (almost all) string types have totally
        ordered values, hence allowing interval definitions. For
        instance:

\begin{verbatim}
PositiveOrZeroInteger ::= INTEGER (0..MAX)
PositiveInteger ::= INTEGER (0<..MAX)
NegativeOrZeroInteger ::= INTEGER (MIN..0)
NegativeInteger ::= INTEGER (MIN..<0)
PositiveReal ::= REAL (0<..PLUS-INFINITY)
NegativeReal ::= REAL (MINUS-INFINITY..<0)
RealInterval ::= REAL (4e-5..1e-4)
\end{verbatim}

%%\continued
\pagebreak

  \item \textsf{Value Constraint}: to restrict the set of values of a
        type to be a singleton:

\begin{verbatim}
Wednesday ::= Day (wednesday)
\end{verbatim}

  \item \textsf{Union Constraint}: the new subtype contains the
        values of the first subtype and of the second subtype (keyword
        \texttt{UNION} or symbol \texttt{|}):

\begin{verbatim}
Day ::= ENUMERATED {monday, tuesday, wednesday, 
                    thursday, friday, saturday, sunday}
WeekEnd ::= Day (saturday | sunday)
\end{verbatim}

  \item \textsf{Alphabet Constraint}: the strings can be restricted to
        be built upon a given alphabet:

\begin{verbatim}
CapitalAndSmall ::= 
  IA5String (FROM ("A".."Z" | "a".."z"))
CapitalOrSmall ::= 
  IA5STring (FROM ("A".."Z") | FROM ("a".."z"))
\end{verbatim}

%%\continued
\pagebreak

  \item \textsf{Size Constraint}: the values of string types may be
        constrained to a given sizes, introducing a constraint by the
        keyword \texttt{SIZE}:

\begin{verbatim}
Exactly31BitsString ::= BIT STRING (SIZE (31))
StringOf31BitsAtTheMost ::= BIT STRING (SIZE (0..31))
NonEmptyString ::= BIT STRING (SIZE (1..MAX))
\end{verbatim}

The size constraint can also apply to \texttt{SET OF} types. In that
case, the semantics is very different: the values of the types are
sets whose \emph{cardinals} are specified by the size constraint:

\begin{verbatim}
SetOf5Strings ::= SET (SIZE(5)) OF PrintableString
SetOfStringsOf5Char::= SET OF PrintableString (SIZE(5))
\end{verbatim}

%%\continued
\pagebreak

  \item \textsf{Intersection Constraint}: the new subtype contains the
        values that belong to the two subtypes (keyword
        \texttt{INTERSECTION} or symbol \texttt{\symbol{94}}):

\begin{verbatim}
FrenchPhoneNumber ::= 
  NumericString (FROM ("0".."9") ^ SIZE (10))
\end{verbatim}

\bigskip\bigskip

  \item \textsf{Inclusion Constraint}: to restrict a subtype to have
        only the values of a given subtype (optional keyword
        \texttt{INCLUDES}):

\begin{verbatim}
LongWeekEnd ::= Day ((INCLUDES (WeekEnd)) | monday)
Bis ::= Day (WeekEnd | monday)
\end{verbatim}

%%\continued 
\pagebreak

  \item \textsf{Complement Constraint}: to restrict the values of a
        type to \emph{not} belong to another subtype:

\begin{verbatim}
Lipogramme ::= IA5String (FROM (ALL EXCEPT ("e" | "E")))
\end{verbatim}

\bigskip\bigskip

  \item \textsf{Constraint on \texttt{SET OF}}: to restrict the
        elements of a \texttt{SET OF} value:

\begin{verbatim}
TextBlock ::= SET OF VisibleString
AddressBlock ::= TextBlock (WITH COMPONENT (SIZE(1..32)))
\end{verbatim}

%%\continued
\pagebreak

  \item \textsf{Partial Constraint}: to restrict \emph{some} fields
        of a \texttt{SET} or \texttt{CHOICE}:

\begin{verbatim}
Quadruple ::= SET {
  alpha ENUMERATED {in, out} OPTIONAL,
  beta  IA5String OPTIONAL,
  gamma SET OF INTEGER,
  delta BOOLEAN DEFAULT TRUE}
\end{verbatim}

we can derive a subtype whose component \verb+alpha+ is always present
and equals \verb+in+, and the component \verb+gamma+ always has
five elements:
\begin{verbatim}
Quadruple1 ::=
  Quadruple (WITH COMPONENTS {..., 
                              alpha (in) PRESENT,  
                              gamma (SIZE (5))})
\end{verbatim}

%%\continued
\pagebreak

\noindent
This subtype has the same values as:
\begin{verbatim}
Quadruple1 ::= SET {
  alpha ENUMERATED {in, out} (in),
  beta  IA5String OPTIONAL,
  gamma SET SIZE (5) OF INTEGER,
  delta BOOLEAN DEFAULT TRUE}
\end{verbatim}

%%  \item \textsf{Complete Constraint}: to restrict \emph{all} fields of
%%        a \texttt{SET} or \texttt{CHOICE}:
%%
%%\begin{verbatim}
%%Quadruple1 ::=
%%  Quadruple (WITH COMPONENTS {alpha (in),
%%                              gamma (SIZE (5)),
%%                              delta})
%%\end{verbatim}
%%
%%\continued
%%
%%\noindent
%%This subtype has the same values as:
%%
%%\begin{verbatim}
%%Quadruple1 ::= SET {
%%  alpha ENUMERATED {in, out} (in),
%%  gamma SET SIZE (5) OF INTEGER,
%%  delta BOOLEAN DEFAULT TRUE}
%%\end{verbatim}

\end{itemize}

\end{slide}

\section{Issues in Validation}
\begin{slide}
\btitle{\textsf{Issues in Validation}}

\ASN types are considered as sets of values, thus types must have at
least one (finite) value.\\

Because of the great expressivity of \ASN, the compilers are not
likely to fully check arbitrary combinations of subtyping
constraints!\\

Vendors' argument: ``This is hardly a real problem because
those critical specifications are rarely found in practice.'' \\

My argument: ``Why not solve completely the problem (for X.680) and
get a better product?''

\bigskip \bigskip \bigskip \bigskip \bigskip

%%\continued
\pagebreak

\textbf{Preliminaries}\\

First, we rewrite every X.680 specification
into a subset of X.680 which has the same expressiveness as X.680,
though having less ambiguities and syntactical constructs (formal
definitions are easier too).\\

Especially, for each value declaration, we extract the given type and
create a corresponding type declaration for it. We also create another
type declaration where the previous type is \emph{constrained} in
order to contain the originally declared value. For instance:

\begin{tabular}{rcl}
   \verb+y INTEGER (0..9) ::= 1+
   & $\longrightarrow$ 
   & $\left\{
       \begin{tabular}{l} 
           \verb+y A ::= 1+ \\
           \verb+A ::= INTEGER (0..9)+ \\
           \verb+B ::= A (y)+
        \end{tabular}
     \right.$
\end{tabular}

where \verb+A+ and \verb+B+ are fresh type references.

%%\continued
\pagebreak

\textbf{Problems}\\

At this stage it can happen that

\begin{enumerate}
  
  \item \label{finiteness} the types may have only infinite
        values, as \texttt{T ::= SET \{a T\}};

  \item \label{type_conformance} some value declarations may be
        ill-typed, as \texttt{v REAL ::= ""};

  \item \label{type_compatibility} especially, some value references
        may be ill-typed, as\\
        \verb+a VisibleString ::= b    b INTEGER ::= 0+;

  \item \label{constraint_consistence} the subtype constraints may
        be inconsistent: \texttt{T ::= REAL (SIZE(7))};

  \item \label{subtype_non_emptyness} the subtypes may be empty, as\\
        \texttt{T ::= SET ((SIZE (1)) \symbol{94} (SIZE (2))) OF REAL};
        
  \item \label{solvability} the subtypes may have no value set:
        \texttt{T ::= REAL (ALL EXCEPT T)}. 

\end{enumerate}

%%\continued
\pagebreak

\vspace*{1cm}

In other words, the issues to settle are respectively:

\begin{enumerate}

  \item the finiteness problem;
 
  \item the typechecking problem;

  \item the type compatibility problem;

  \item the constraint consistence problem;

  \item the non-emptyness problem;

  \item the solvability problem.

\end{enumerate}

%%\continued
\pagebreak

\textbf{Problem analysis}

\begin{itemize}

  \item type compatibility $\subseteq$ typechecking;

  \item constraint consistence, non-emptyness $\subseteq$ solvability
        (the system is solved when we construct explicitly the values
        of each subtype);

  \item typechecking $\subseteq$ solvability (we added initially a new
        type declaration for each value declaration).

\end{itemize}

So, finiteness and solvability are enough to get a full validation of
X.680.

\end{slide}



\section{A Constraint-based Solution}
\begin{slide}
\btitle{A Constraint-based Solution}

\textbf{Well-founded Types}\\

A \emph{well-founded type} is a type uniquely defined and which has at
least one finite value.\\

We give an algorithm to check whether a type is
well-founded. Basically, it analyses precisely the recursions.\\

Note: the well-foundedness of a type does \emph{not} imply the
well-foundedness of its subtypes:

\begin{verbatim}
  T ::= CHOICE {a T, b REAL}
  U ::= T (WITH COMPONENTS {..., b ABSENT})
\end{verbatim}

type \verb+T+ is well-founded but \verb+U+ is not.

%%\continued
\pagebreak

\textbf{Set Constraints}\\

Sets constraints are inclusions between expressions interpreted over
the domain of sets of trees which may be recursively defined.\\

The \ASN types, by having a tree structure (technically, they are
rational trees), fit perfectly in the usual domain of set
constraints.\\

A set constraint has the form $\alpha \subseteq \beta$, where $\alpha$
and $\beta$ are \emph{set expressions}. Examples are \textsf{0} (the
empty set), \textsf{1} (the set of all terms), $\alpha$ (a set-valued
variable), $c(\alpha, \beta)$ (a constructor application), and the
union, intersection or complement of set expressions. Given a set of
constructors ${\cal C}$, where each $c \in {\cal C}$ has arity $a(c)$,
set expressions are defined by the grammar:

$$\E ::= \textsf{0} \mid \textsf{1} \mid \alpha \mid c (\E_1, \ldots,
\E_{a(c)}) \mid \E_1 \cup \E_2 \mid \E_1 \cap \E_2 \mid \neg \E_1$$

\pagebreak

\textbf{Common Application}\\

Set constraints are used in program analysis: sets of terms describe
the possibly computed values. Set constraints are generated from the
program text and solving the former yields some useful information
about the latter (e.g. type-ckecking or optimisation).\\

For example, consider a program with two uses of a variable
$v$. Assume that from one use we determine that $v$ must be a number
(i.e. $v \subseteq \textsf{Int} \, \cup \, \textsf{Float}$) and from
the other use that $v$ cannot be an integer (i.e. $v \subseteq
\neg\textsf{Int}$). Combining these constraints, we can infer that $v$
must be a floating point number (i.e. $v \subseteq (\textsf{Int} \,
\cup \, \textsf{Float}) \cap \neg\textsf{Int} = \textsf{Float}$).

\pagebreak

\textbf{Application to \ASN}\\

In order to apply the set constraints to \ASN we only define specific
constructors $c$ that model all the subtyping constraints:

\bigskip

\ocamlkwd{type} \ocamltypename{construct}$_{1}$ = 
  \ocamlconstr{[}\Null{} $\mid$ \kwdTRUE{} $\mid$
  \kwdFALSE\ocamlconstr{]}

\noindent
\ocamlkwd{and} \ocamltypename{construct}$_{2}$ = \\
\hspace*{3mm}
\ocamlconstr{[}\Enum{} \ocamlkwd{of} \ocamltypename{string}
$\mid$ \Regexp{} \ocamlkwd{of} \ocamltypename{string}
$\mid$ \ocamltypename{integer}
$\mid$ \ocamltypename{real}
$\mid$ \ocamltypename{construct}$_{1}$\ocamlconstr{]}

\noindent
\ocamlkwd{and} \ocamltypename{real\_interval} =\\
\hspace*{3mm} 
\ocamlconstr{[}\wildcard \wildcard \asnkwdconstr{..} \wildcard
\wildcardof \ocamltypename{real\_bound} $\times$ \ocamltypename{in\_out} 
$\times$ \ocamltypename{in\_out} $\times$ \ocamltypename{real\_bound}\ocamlconstr{]}

\noindent
\ocamlkwd{and} \ocamltypename{real\_bound} =
\ocamlconstr{[}\ocamltypename{real} 
$\mid$ \MinInfReal{} 
$\mid$ \PlusInfReal \ocamlconstr{]}

\noindent
\ocamlkwd{and} \ocamltypename{closed\_int\_interval} = 
\ocamlconstr{[}\Interval{} \ocamlkwd{of} \ocamltypename{int\_bound}
$\times$ \ocamltypename{int\_bound}\ocamlconstr{]}

\noindent
\ocamlkwd{and} \ocamltypename{int\_bound} =
\ocamlconstr{[}\ocamltypename{integer}{} 
$\mid$ \MinInfInt{}
$\mid$ \PlusInfInt \ocamlconstr{]}

\smallskip

\noindent
\ocamlkwd{and} \ocamltypename{list} = 
\ocamlconstr{[}\Cons{} \ocamlkwd{of} $\E$ $\times$ \ocamltypename{list}
$\mid$ \Nil \ocamlconstr{]}

\noindent
\ocamlkwd{and} \ocamltypename{map} =
\ocamlconstr{[}\Bind{} \ocamlkwd{of} \ocamltypename{identifier}
$\times$ $\E$ $\times$ \ocamltypename{map} $\mid$ \Nil \ocamlconstr{]}

\smallskip

\noindent
\ocamlkwd{and} \ocamltypename{construct} =
       \ocamlconstr{[}\ocamltypename{list} 
$\mid$ \ocamltypename{map}
$\mid$ \wildcard\/ $\pmb{:}$ \wildcardof
           \ocamltypename{label} $\times$ $\E$\\
\hspace*{3mm}
$\mid$ \ocamltypename{real\_interval}
$\mid$ \ocamltypename{closed\_int\_interval}
$\mid$ \ocamltypename{construct}$_{2}$\ocamlconstr{]}

\pagebreak

As a special case, we define some useful set constants:

\medskip

\noindent
$\ocamlkwd{let} \,\, \mathbb{N} = \Interval \, (\MinInfInt,
\PlusInfInt)$\\
$\ocamlkwd{and} \,\, \mathbb{N}^{+} = \Interval \, (\PosInt \, (0),
\PlusInfInt)$\\
$\ocamlkwd{and} \,\, \mathbb{R} = \MinInfReal \, \pmb{<}
\asnkwdconstr{..} \pmb{<} \, \PlusInfReal$

\bigskip\bigskip\bigskip

Now the problem is to extract set constraints from each subtype, such
as their solving provide a mapping from the subtype to its set of
values: this is a \emph{denotational semantics} of \ASN.\\

As far as validation is concerned, if some constraints are unsolvable
or the set of solutions is empty, them the specification must be
rejected.

\pagebreak

\btitle{Constraint Collection}

\textbf{From Types}\\

\begin{tabular}{rcl}
$\llbracket{\kwdINTEGER}\rrbracket_{\alpha}$
  & = & $(\alpha \CEeq \mathbb{N})$\\ 
$\llbracket{\kwdREAL}\rrbracket_{\alpha}$
  & = & $\alpha \CEeq \mathbb{R}$\\ 
$\llbracket{\kwdBOOLEAN}\rrbracket_{\alpha}$
  & = & $\alpha \CEeq \kwdTRUE \,\, \SEcup \,\, \kwdFALSE$\\
$\llbracket{\kwdENUMERATED \, [a]}\rrbracket_{\alpha}$
  & = & $\alpha \CEeq \Enum \, (a)$\\
$\llbracket{\kwdENUMERATED \, (a \CONS b \CONS
  \textnormal{I})}\rrbracket_{\alpha}$
  & = & $\ocamlkwd{let} \,\, \beta \,\, \textnormal{\emph{be a fresh
         variable}}$\\
  &   & $\ocamlkwd{in} \,\, \llbracket{\kwdENUMERATED \, (b \CONS
        \textnormal{I})}\rrbracket_{\beta}$ \\
  &   & \hspace*{8mm}
        $\wedge \,\,\,  \alpha \CEeq \Enum \, (a) \, \SEcup \, \beta$\\
\end{tabular}

\bigskip

\emph{et cetera}. \\


\textbf{From Subtypes}\\

Similarly, constraints are collected from (proper) subtypes.


\pagebreak

\btitle{Constraint Solving}

Aiken and Wimmers proposed the first solving algorithm for the class
of constraints we are using: \\

\bibliographystyle{plain}
\bibliography{asn1_semantics_talk}
\nocite*{}

\end{slide}

\section{Conclusion}
\begin{slide}
\btitle{Conclusion}

\begin{itemize}

  \item First formal and complete semantics of \ASN (X.680). It is
        complementary to the usual syntax-driven informal approach.

  \item First algorithm for the full validation of \ASN (X.680).

  \item Full validation of \ASN means full interoperability at the
        data level between network equipments.

  \item The algorithm is given in a meta-language whose semantics
        already exists and there is a programming language for it
        (namely Objective Caml, cf.
        \url{http://www.ocaml.org} and \url{http://caml.inria.fr/}).

  \item Best site about \ASN: \url{http://asn1.elibel.tm.fr/}

\end{itemize}

\end{slide}



\end{document}
