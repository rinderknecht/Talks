\section{Annexes}
\begin{slide}
\begin{block} {Annexes} \end{block}
\end{slide}


\begin{slide}
\begin{block} {Packed Encoding Rules (PER)} \end{block}

\begin{itemize}
  \item Le but est de r�duire l'information de contr�le (ie. de type)
        dans les codes.

  \item Philosophie des PER
        \begin{itemize}
           \item Tri canonique des champs de structures en fonction de
                 leurs �tiquettes.
           \item Certains codes ont un {\em bit-map} pour g�rer les
                 �ventuels champs optionnels et par d�faut. 
           \item Utilisation des contraintes de sous-typage pour
                 obtenir des codes plus courts (ex. valeurs de {\tt
                 INTEGER (0|1)} cod�es avec un bit).
        \end{itemize}
\end{itemize}

\continued

\begin{itemize}
  \item Avantage: Le ratio ``contenu utile/contr�le'' tr�s grand.

  \item Inconv�nients:
        \begin{itemize}
           \item Norme tr�s complexe par rapport aux BER, et donc mal
                 comprise, m�me par les membres de l'ISO.

           \item Beaucoup plus de r��critures sur les sp�cifications et
                 plus d'hypoth�ses pour le d�codage (par rapport aux
                 BER). 
        \end{itemize}
\end{itemize}

\end{slide}




\begin{slide}
\begin{block} {Usages d'ASN.1} \end{block}
\begin{itemize}
  \item HTTP-Next Generation
  \item CMIP (Network administration)
  \item RNIS (Digital phone services)
  \item ATM (Asynchronous Transfert Mode)
  \item X.400 (Message handling system)
  \item X.500 (Phone repository)
  \item MHEG, T.120, H.245 (Multimedia conferencing)
  \item Switching systems (Routers)
  \item Mobile phones (Connection, switching)
  \item SDL, Estelle (Protocol specification languages)
  \item TTCN (Protocol testing language)
  \item Delivery tracking (Federal Express)
  \item World-wide financial transactions (Reuter)
  \item Reactive power control in electric utilities
\end{itemize}

\begin{itemize}
  \item Air-ground \& ground-ground comm (KLM, UA, International Civil
Aviation Organization)
  \item Power quality monitoring over the Internet
  \item Diagnostic monitoring systems in auto plants
  \item Future printer job management (DEV, HP, IBM, Sun, Xerox)
  \item Subway of Berlin
  \item VHDL (Electronic circuit design language)
  \item NCBI (Genetic database)
  \item Office Automation (EDI, ODA)
  \item Image processing (IPI)
  \item Brewing of beer
\end{itemize}
\end{slide}




\begin{slide}
\begin{block} {L'�quivalence de valeurs (exemple de r�gle)} \end{block}

\deduction 
  \Phi \lhd [\textsf{Field} (\textrm{label}, \textrm{T}')] \sqcup \Phi'
  \cr
  V_0 \lhd [(\textrm{label}, v'_0)] \sqcup V'_0
  \cr
  V_1 \lhd [(\textrm{label}, v'_1)] \sqcup V'_1
  \cr
  \Gamma, \textrm{T}'\vdash v'_0 \sim v'_1
  \cr
  \Gamma, \textsf{SET} (\Phi') \vdash \bob V'_0 \bcb \sim \bob V'_1 \bcb
  ------------------------------------------------------------------------
  \Gamma, \textsf{SET} (\Phi) \vdash \bob V_0 \bcb \sim \bob V_1 \bcb
\end

\end{slide}




\begin{slide}
\begin{block} {Pr�sentation d'ASN.1} \end{block}

\noindent {\bf Types de base}

\begin{itemize}
  \item 
\begin{verbatim}
NumberOfCustomers ::= INTEGER
current NumberOfCustomers ::= 1000000
\end{verbatim}

  \item
\begin{verbatim}
DayInTheYear ::= INTEGER {first(1), last(356)}
newYearsEve DayInTheYear ::= last
\end{verbatim}


  \item
\begin{verbatim}
Synchro-Indicator ::= ENUMERATED {serial(1), parallel(2)}
synchro Synchro-Indicator ::= serial
\end{verbatim}


  \item
\begin{verbatim}
SwitchStatus ::= BOOLEAN
\end{verbatim}


  \item
\begin{verbatim}
padding NULL ::= NULL
\end{verbatim}

\end{itemize}

\end{slide}

\begin{slide}
\begin{block} {Types Construits} \end{block}

\begin{itemize}
  \item {\bf Construction par r�p�tition}
        \begin{itemize}
          \item 
\begin{verbatim}
A ::= SET OF INTEGER
empty A::= {}
small A ::= {7, 9, 1, 1, 3}
\end{verbatim}

          \item 
\begin{verbatim}
B ::= SEQUENCE OF INTEGER
\end{verbatim}
        \end{itemize}

  \item {\bf Construction par agr�gation} 
        \begin{itemize}
\item
\begin{verbatim}
PersonInfo ::= SET {
                 age     INTEGER,
                 married BOOLEAN}
i PersonInfo ::= {married FALSE, age 28}
\end{verbatim}

  \item 
\begin{verbatim}
PersonInfo ::= SEQUENCE {
                 age     INTEGER,
                 married BOOLEAN}
i PersonInfo ::= {married FALSE, age 28} -- KO!
i PersonInfo ::= {age 28, married FALSE} -- OK!
\end{verbatim}
        \end{itemize}

  \item 
\begin{verbatim}
Coordinates ::= SEQUENCE {x REAL, y REAL}
origin Coordinates ::= {x 0, y 0}
\end{verbatim}

  \item 
\begin{verbatim}
Coordinates ::= SET {x REAL, y REAL}
\end{verbatim}                         

Comment le r�cepteur distinguera les deux champs? Avec les {\em
�tiquettes}:

\begin{verbatim}
Coordinates ::= SET {x [0] REAL, y [1] REAL}
\end{verbatim}

  \item Tous les champs de {\tt SET} doivent avoir des �tiquettes
        distinctes. 

  \item L'�tiquetage n'affecte pas la syntaxe des valeurs.
\end{itemize}

\end{slide}

\begin{slide}
\begin{itemize}
  \item 
\begin{verbatim}
Coordinates ::= SEQUENCE {
                  x REAL DEFAULT 0, 
                  y REAL DEFAULT 0}
origin Coordinates ::= {}
\end{verbatim}
...est �quivalent �: 
\begin{verbatim}
origin Coodinates ::= {x 0, y 0}
\end{verbatim}

  \item
\begin{verbatim}
DataAcknowledgementTPDU ::= SET {
  destRef        [0] Reference,
  yr-tu-nr       [1] TPDUnumber,
  checkSum       [2] CheckSum OPTIONAL,
  subSeqNr       [3] SubSequenceNumber DEFAULT 0,
  flowControlCnf [4] FlowControlConfirmation OPTIONAL
}
\end{verbatim}

\end{itemize}
\end{slide}

\begin{slide}
\begin{itemize}
  \item
\begin{verbatim}
Illegal ::= SEQUENCE {
              x REAL OPTIONAL,
              y REAL
            }
\end{verbatim}
  \item Une suite de champs {\tt OPTIONAL} et le champ suivant
        doivent avoir des �tiquettes distinctes.
\end{itemize}
\end{slide}

\begin{slide}
\begin{itemize}
  \item {\bf Construction par union} \\
        (semblable � {\tt union} du langage C)
\begin{verbatim}
T ::= CHOICE {
        x REAL,
        y INTEGER,
        z BOOLEAN
      }
v1 T ::= x : {50, 10, -1}
v2 T ::= z : TRUE
\end{verbatim}

  \item 
\begin{verbatim}
Illegal ::= CHOICE {
              a CHOICE {aa INTEGER, ab BOOLEAN},
              b CHOICE {ba BOOLEAN, bb NULL}
            }
\end{verbatim}

  \item Tous les champs de {\tt CHOICE} doivent avoir des �tiquettes
        distinctes, en parcourant en profondeur le type.
\end{itemize}
\end{slide}


\begin{slide}
\begin{block} {R�cursivit�} \end{block}
\begin{itemize}
  \item Les types r�cursifs sont autoris�s et... probables:
\begin{verbatim}
CMISFilter ::= CHOICE {
  item  [8] FilterItem,
  and   [9] IMPLICIT SET OF CMISFilter,
  or   [10] IMPLICIT SET OF CMISFilter,
  not  [11] CMISFilter
}
\end{verbatim}

  \item Les valeurs r�cursives sont ill�gales (sur notre proposition �
        l'ISO). 
\end{itemize}
\end{slide}

\begin{slide}
\begin{block} {Sous-typage} \end{block}

\begin{itemize}

  \item 
\begin{verbatim}
PositiveInt ::= INTEGER (0<..MAX)
PositiveReal ::= REAL (0<..PLUS-INFINITY)
Small-primes ::= INTEGER (2 | 3 | 5 | 7)
\end{verbatim}

\item
\begin{verbatim}
Word-32bits ::= BIT STRING (SIZE (0..31))
Morse ::= PrintableString (FROM ("." | "-" | " "))
Social-Insurance ::= 
       NumericString (SIZE (13)) (FROM ("0".."9"))
\end{verbatim}

  \item
\begin{verbatim}
A ::= SET OF PrintableString (SIZE (5))
B ::= SET (SIZE (5)) OF PrintableString
\end{verbatim}
\end{itemize}

\continued

\begin{itemize}

  \item 
\begin{verbatim}
C ::= SET {
        a INTEGER,
        b NumericString
      } (WITH COMPONENTS {..., a (0|1)})
\end{verbatim}

...est �quivalent �:

\begin{verbatim}
C ::= SET {
        a INTEGER (0|1),
        b NumericString
      }
\end{verbatim}

\end{itemize}

\end{slide}

