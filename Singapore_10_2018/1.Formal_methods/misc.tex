%%-*-latex-*-

\documentclass[wide]{slides}

% Language
%
\usepackage{hyphenat}          % \hyp{} is a breakable dash

\frenchspacing  % Follow French conventions after a period

% Maths & Logic
%
\usepackage{stmaryrd}
\usepackage{mathpartir}

% Input files
%
../TeX/commands.tex

% ----------------------------------------------------------------
% Document
%
\maintitle{Formal Methods in Software Engineering II}
\mainauthor{\textbf{Christian Rinderknecht}\\
  {\small\url{Christian.Rinderknecht@tezcore.com}}\\
\Nomadic}
\confname{Tezos Masterclass}
\confshortname{Tezos}
\confdate{19 October 2018}

\begin{document}

\maketitle


\begin{slide}
  \title{Formal logic}

  The relevance and importance of logic in the context of software
  engineering consists of two aspects:
  \begin{enumerate}

    \item logic provides a framework to express sundry notions,

    \item logic lends itself well to formalisation.

  \end{enumerate}

\end{slide}

\begin{slide}
  \title{Logic as a framework}

  As far as the first aspect is concerned,
  \begin{itemize}

    \item the data processed by programs can be described as
      elementary sets (integers, characters etc.) being combined, for
      example by means of Cartesian products (e.g., the records of
      \OCaml) or unions (e.g., the variant types of
      \OCaml).

    \item Computability theory informs us of the existence of limits,
      that is, of specifications that can be expressed but not
      satisfied.

    \item Type theory leads to compilers that ensure the safety of
      programming by matching given type annotations against the
      expressions, and/or by inferring those types from the
      expressions.

  \end{itemize}

\end{slide}

\begin{slide}
  \title{Logic for modelling}

  As far as the second aspect is concerned, there are two reasons for
  formalising:
  \begin{itemize}

    \item the rigour of the texts and the reasonings is enhanced, for
      they can be reduced, at least in part, to some interpretative
      processing of symbols;

    \item the effort can be alleviated by the use of interactive tools
      or even, sometimes, some tasks can be completely automated.
  \end{itemize}

\end{slide}

\begin{slide}
  \title{Theorem proving}

  \begin{itemize}

    \item Another approach is \textbf{theorem proving}, which delivers
      the same guarantees as model checking (except in presence of a
      counterexample) but does not require the model to be finite,
      because it relies on the principle of \textbf{induction}.

    \item Some proof assistants, like \Coq (implemented in \OCaml),
      can even generate implementations from proofs, thanks to the use
      of a constructive logic.

    \item Theorem proving can fruitfully be used in conjunction with
      model checking.

  \end{itemize}
\end{slide}

\begin{slide}
  \title{Static analysis}
  \begin{itemize}

    \item Finally, \textbf{static analysis} consists in approximating
      sets of values and behaviours of implementations as they would
      occur at run-time, without instrumenting the source code, nor
      actually running it.

    \item The need for approximations stems both from theoretical
      reasons (undecidable problems), and practical issues (state
      explosion).

    \item A powerful theoretical framework is \textbf{abstract
      interpretation}, which is based on a sound abstraction of the
      semantics of a programming language and a symbolic execution of
      the program.

    \item More syntactical algorithms (therefore less powerful), like
      \textbf{type checking} or \textbf{type inference}, also work
      well in practice and both approaches are useful.

    \item Static analysis can be thought of as abstracting a model
      from a program, and is thus complementary to the techniques we
      have mentioned earlier.

  \end{itemize}
\end{slide}

\begin{slide}
  \title{Abstract interpretation}

  \begin{itemize}

    \item Abstract interpretation can be used to analyse systems whose
      number of possible states is infinite.

    \item A \textbf{state} is all the information characterising a
      system running at a given point in time (memory contents,
      program counter, files, network buffers, etc.)

    \item A behaviour of a system, or \textbf{trace}, can be thought
      as a line from one state to the next. The (usually) infinite set
      of traces is the \textbf{concrete semantics} of the program.

    \item Abstract interpretation over\hyp{}approximates the states
      and behaviours of a system, yielding, respectively, an
      \textbf{abstract domain} and an \textbf{abstract semantics}.

    \item Concrete properties to be checked are abstracted in such a
      way that abstract states satisfying the property can be
      discovered by a \textbf{terminating interpretation of the
        abstract program}, and translated each back to sets of
      concrete states.

  \end{itemize}

\end{slide}

\begin{slide}
  \title{Abstract interpretation}

  \begin{itemize}

    \item Like all static analyses, abstract interpretation is an
      over\hyp{}approximation, which means that it may find an
      erroneous abstract state whose corresponding concrete states are
      \emph{not} all erroneous, but it cannot miss concrete erroneous
      states (it is a \textbf{sound approximation}).

    \item To reduce the number of those \textbf{false positives}, the
      abstract semantics must become closer to the concrete semantics,
      whilst the property remaining tractable (that is, both
      theoretically and practically computable). In other words, we
      seek a smaller over\hyp{}approximation.

    \item The abstract states should retain enough information to
      correspond to useful sets of concrete states.

    \item For instance, if the concrete domain is the set of integers,
      a possible abstract domain could be the signs of the integers
      (little information), or their remainders by~\(16\) (more
      information), etc.

  \end{itemize}

\end{slide}

\begin{slide}
  \title{Abstract interpretation / An illustration}

  \begin{itemize}

    \item As an illustration, let us consider a system whose states
      are pairs of positive real numbers.

    \item Let us choose an abstract domain made of pairs of natural
      numbers: the nearest integral coordinate represents a set of
      concrete states.

    \item Some abstract states are coloured in \textcolor{red}{red} to
      denote that they are erroneous: this is the abstract property
      ``To be erroneous.''

    \item We show four abstract traces as \textbf{black} lines. (They
      potentially denote an infinite set of concrete traces, all
      traces making up the abstract semantics itself.)

    \item One of the abstract traces actually ends in an error (but it
      may be that not all the corresponding concrete states are
      erroneous).

  \end{itemize}

\end{slide}

\begin{slide}
  \title{Abstract interpretation / An illustration}

  \begin{center}
    \includegraphics[bb=95 428 500 727, scale=0.64]{traces.eps}
  \end{center}

\end{slide}

\begin{slide}
  \title{Abstract interpretation / An illustration}

  \begin{itemize}

    \item In general, the number of erroneous states in a finite
      region can be infinite (imagine that the abstract states are
      rational numbers, instead of natural numbers).

    \item We need to characterise the erroneous states in a finite
      manner, at the cost of over\hyp{}approximating them: this means
      to abstract the property ``To be erroneous'' so it becomes
      easier to check.

    \item For example, we can abstract the error property by a red
      square containing the set of erroneous states, instead of being
      the set itself.

    \item A square on the abstract domain can be defined by three
      numbers only and it is easy to check whether a given state
      belongs to it: if so, we deem it an error.

    \item In our example, this creates false positives, that is,
      states inside the square that are not actual errors.

  \end{itemize}

\end{slide}

\begin{slide}
  \title{Abstract interpretation / An illustration}

  \begin{center}
    \includegraphics[bb=95 428 500 727, scale=0.64]{traces.eps}
  \end{center}

\end{slide}

\begin{slide}
  \title{Abstract interpretation / An illustration}

  \begin{center}
    \includegraphics[bb=95 428 500 727, scale=0.64]{traces_square.eps}
  \end{center}

\end{slide}

\begin{slide}
  \title{Abstract interpretation / An illustration}

  \begin{center}
\includegraphics[bb=95 428 500 727,scale=0.64]{traces_square_cropped.eps}
  \end{center}

\end{slide}

\begin{slide}
  \title{Abstract interpretation / An illustration}

  \begin{itemize}

    \item To reduce the number of false positives, we \textbf{refine
      the abstract property} so it gets closer to the concrete
      property.

    \item For instance, we could use another polygon to cover more
      tightly the erroneous states.

    \item Let us try an hexagon: to define it, we need the same amount
      of information as for a square, but the inclusion property is
      slightly more complex.

    \item This is better, but one trace remains a false positive.

    \item In general, false positives are inevitable if the property
      is complex enough, so trade\hyp{}offs are necessary.

  \end{itemize}

\end{slide}

\begin{slide}
  \title{Abstract interpretation}

  \begin{center}
    \includegraphics[bb=85 428 500 727, scale=0.64]{traces_square_cropped.eps}
  \end{center}

\end{slide}

\begin{slide}
  \title{Abstract interpretation}

  \begin{center}
    \includegraphics[bb=85 428 500 727, scale=0.64]{traces_hexagon.eps}
  \end{center}

\end{slide}

\begin{slide}
  \title{The proof assistant Coq: An example}

  \begin{itemize}

    \item To give a taste of what can \Coq automatically do for us,
      let us consider a microscopic example.

    \item We are going to need a simple data structure called a
      \textbf{stack}.

    \item A stack can be thought of as a finite series of items that
      can only be accessed sequentially from one end, called the
      \textbf{top}, following the analogy with a stack of material
      objects.

    \item We need to define all the shapes that a stack can take. A
      stack can either be empty or contain at least one item.

    \item Let us denote the empty stack by ``\(\fun{nil}\)'', and the
      stack with \(x\)~on top of the stack~\(s\) as \(\cons{x}{s}\).

    \item For example, the stack containing only~\(0\) is noted
      \(\cons{0}{\fun{nil}}\), the stack containing~\(1\) on top,
      then~\(0\) is \(\cons{1}{\cons{0}{\fun{nil}}}\).

    \item The operator (\(::\)) associates to the right, so
      \(\cons{x}{\cons{y}{s}}\) actually means
      \(\cons{x}{(\cons{y}{s})}\).

  \end{itemize}
\end{slide}

\begin{slide}
  \title{An short example in Coq}

  \begin{itemize}

    \item Let us define a \textbf{binary relation} (\(\approx\)) over
      stacks that capture the notion of \textbf{permutation}.

    \item Intuitively, the permutation of a stack is another stack
      containing the same items, but not necessarily in the same
      order.

    \item For example, ``\(\fun{nil}\)'' is a trivial permutation of
      ``\(\fun{nil}\)'', and
      \(\cons{2}{\cons{1}{\cons{0}{\fun{nil}}}}\) is a permutation of
      \(\cons{0}{\cons{2}{\cons{1}{\fun{nil}}}}\), formally:
      \begin{itemize}

        \item \(\fun{nil} \approx \fun{nil}\),

        \item \(\cons{2}{\cons{1}{\cons{0}{\fun{nil}}}} \approx
          \cons{0}{\cons{2}{\cons{1}{\fun{nil}}}}\).

      \end{itemize}

    \item The relation (\(\approx\)) can be defined in different
      ways.

    \item The idea here consists in defining a permutation as a series
      of \textbf{transpositions}, namely, exchanges of adjacent items.

  \end{itemize}

\end{slide}

\begin{slide}
  \title{An short example in Coq}

  Here is an \textbf{inductive definition} of permutations, based on
  transpositions:
  \begin{enumerate}

    \item \label{perm:nil} \(\fun{nil} \;\approx\; \fun{nil}\),

    \item \label{perm:swap} \(\cons{x}{\cons{y}{s}} \;\approx\;
      \cons{y}{\cons{x}{s}}\),

    \item \label{perm:push} if \(s \;\approx\; t\), then \(\cons{x}{s}
      \;\approx\; \cons{x}{t}\),

    \item \label{perm:trans} if \(s \;\approx\; u\) and \(u
      \;\approx\; t\), then \(s \;\approx\; t\). (This is the
      \textbf{transitivity} property.)

  \end{enumerate}
  We are going to use a special notation to write the same definition,
  called \textbf{inference systems}:
  \begin{mathpar}
    \inferrule*{}{\el \;\approx\; \el}
    \;\TirName{Nil}
    \qquad
    \inferrule*{}{\cons{x}{\cons{y}{s}} \;\approx\; \cons{y}{\cons{x}{s}}}
    \;\;\TirName{Swap}\\
    \inferrule
        {s \;\approx\; t}
        {\cons{x}{s} \;\approx\; \cons{x}{t}}
        \;\TirName{Push}
        \qquad
        \inferrule
            {s \;\approx\; u \and u \;\approx\; t}
            {s \;\approx\; t}
            \;\TirName{Trans}
  \end{mathpar}

\end{slide}

\begin{slide}
  \title{Inference systems}

  \begin{itemize}

    \item Each case is called an \textbf{inference rule}.

    \item Beyond removing potientially ambiguous English statements
      from the definition, what makes inference systems interesting is
      that they enable two kinds of interpretations: logical and
      computational.

    \item The \textbf{logical interpretation} considers inference
      rules as logical implications of the form \(P_1 \wedge P_2
      \wedge \ldots \wedge P_n \;\Rightarrow\; C\), written
      \begin{equation*}
        \inferrule
            {P_1 \\ P_2 \\ \ldots \\ P_n}
            {C}
      \end{equation*}

    \item The propositions~\(P_i\) are called \textbf{premises} and
      \(C\)~is the \textbf{conclusion}.

    \item In the case of \TirName{Push}, there is only one
      premise. There are two for \TirName{Trans}. When premises are
      lacking, as in \TirName{Nil} and \TirName{Swap}, then \(C\)~is
      called an \textbf{axiom} and no horizontal line is drawn.

  \end{itemize}

\end{slide}

\begin{slide}
  \title{Inference rules}

  \begin{itemize}

    \item The logical interpretation, also called \textbf{deductive},
      is top\hyp{}down.

    \item The \textbf{computational interpretation}, called
      \textbf{inductive} in some contexts, is bottom\hyp{}up: given a
      statement matching the conclusion of a rule, computing that
      statement requires the computation of the premises.

    \item Let us imagine that we want to prove that
      \begin{enumerate}

        \item \(s \;\approx\; s\) (\textbf{reflexivity}),

        \item \(s \;\approx\; t \;\Rightarrow\; t \;\approx\; s\)
          (\textbf{symmetry}).

      \end{enumerate}

    \item In a later lecture, we will do it by hand.

    \item For the moment, here is the \textbf{proof script} for \Coq,
      which is enough to automatically establish these two properties
      by means of tactics, that is, predefined patterns of proof that
      can be composed using heuristics.

  \end{itemize}

\end{slide}

\begin{slide}
  \title{A proof in Coq}
\begin{verbatim}
Set Implicit Arguments.
Require Import List.
Variable A: Type.

Inductive perm: list A -> list A -> Prop :=
  Pnil  : perm nil nil
| Push  : forall x s t, perm s t -> perm (x::s) (x::t)
| Swap  : forall x y s, perm (x::y::s) (y::x::s)
| Trans : forall s t u, perm s u -> perm u t -> perm s t.

Hint Constructors perm.

Lemma reflexivity: forall s, perm s s.
Proof. induction s; eauto. Qed.
Lemma symmetry: forall s t, perm s t -> perm t s.
Proof. induction 1; eauto. Qed.
\end{verbatim}

\end{slide}

\begin{slide}
  \title{Verification of protocols}

  \begin{itemize}

    \item Depending on the scale and the nature of the software
      applications, different network topologies, routing and
      synchronisation protocols may be used, for example centralised
      architectures (client\hyp{}server) or decentralised ones, like
      peer\hyp{}to\hyp{}peer (P2P) or mobile ad hoc networks (MANET),
      where nodes are also routers.

    \item Whether there is an actual network is not essential here:
      more generally, what matters is to have \textbf{independent
        processes} whose interactions are valid according to a
      \textbf{protocol}.

    \item Nowadays, blockchains may feature layered protocols. For
      instance, in the case of the \Tezos blockchain: a network
      protocol, a transaction protocol and a consensus protocol.

    \item These protocols interact in very complex ways and the stakes
      are high to avoid hacks or errors, like unwanted \textbf{feature
        interactions}.

    \item Formal methods can contribute to protocol engineering, as
      the telecommunication industry is already aware.

  \end{itemize}

\end{slide}

\begin{slide}
  \title{Model checking with SPIN}

  \begin{itemize}

    \item \SPIN is a system supporting the design, simulation
      and model checking of \textbf{asynchronous systems}.

    \item It is the most widely used in industry, and the first
      designs and implementations were made at Bell Labs by Holzmann
      in the 80s.

    \item Sequential computations tend to be abstracted so the state
      of the system reflects mainly process interactions.

    \item These can be classified into several kinds:
      \begin{itemize}

        \item asynchronous message passing through buffered channels,

        \item rendezvous primitives (binary synchronisation via
          handshake by means of a channel of buffer size zero),

        \item shared variables,

      \end{itemize}
      and any combination of these.

  \end{itemize}

\end{slide}

\begin{slide}
  \title{Promela}

  \begin{itemize}

    \item The language for defining the models is \Promela (process
      meta\hyp{}language), from which the name \SPIN is actually
      derived: ``Simple \Promela Interpreter.''

    \item It is an imperative specification language with a \Clang
      look and feel, except for the lack of pointers, dynamic memory
      allocation and functions returning values.

    \item The purpose of those restrictions, and others, is mainly to
      constrain the number of states in the model to be
      \textbf{finite}.

    \item In theory, this enables all correctness properties to become
      decidable, although, in practice, some engineering methods and
      techniques are needed to overcome resource limitations in the
      presence of very large state spaces.

  \end{itemize}
\end{slide}

\begin{slide}
  \title{Linear Temporal Logic}

  \begin{itemize}

    \item The language used in \Promela to express temporal formulas
      is based upon the standard \textbf{Linear Temporal Logic} (LTL).

    \item Formulas are interpreted on each execution of a program and
      quantification is always universal, that is, it is not possible
      to isolate a strict subset of paths with respect to a property.

    \item The typical temporal operators used are
      \begin{itemize}

        \item $\bigcirc \varphi$, meaning that $\varphi$ is true in
          the \textbf{next} moment in time;

        \item $\Box \varphi$, meaning that $\varphi$ is true in
          \textbf{all} future moments;

        \item $\lozenge \varphi$, meaning that $\varphi$ is true in
          \textbf{some} future moment;

        \item $\varphi \mathrel{\textsf{U}} \psi$, meaning that
          $\varphi$ is true \textbf{until} $\psi$ is true.

      \end{itemize}
  \end{itemize}
\end{slide}

\begin{slide}
  \title{Linear Temporal Logic}

  \begin{itemize}

    \item The usual connectors of non\hyp{}temporal logic are present
      as well in the LTL, like conjunction, negation, etc. but time is
      \textbf{not} explicitly present, for example in the guise of
      clocks.

    \item As a consequence, time constraints, like \textbf{timeouts},
      can not be modelled, but partial orderings of process
      interactions can.

    \item Once a \Promela model and temporal formulas in LTL have been
      written, \SPIN generates a dedicated \textsf{C}~program, whose
      run checks the claims against the specification.

  \end{itemize}

\end{slide}

\begin{slide}
  \title{Trade-offs}

  \begin{itemize}

    \item In case the state space is too large, some parts of the
      model used for verification have to over\hyp{}approximated.

    \item Also, \Promela allows for the embedding of
      \textbf{\textsf{C}~code as atomic sections} whose scope is local
      and invisible to other processes (note: there is no hierarchy of
      processes).

    \item Of course, this is only feasible if the code in question
      does not deal directly with the control aspect of the protocol,
      otherwise its effects have to be observable in the global state
      space. Purely computational tasks can be abstracted this way,
      though.

    \item If a system features a coupling between a real\hyp{}time
      controller and a continuous physical environment (sensors), it
      is necessary to model the closed\hyp{}loop system as a whole.

  \end{itemize}

\end{slide}

\begin{slide}
  \title{Timed automata}

  \begin{itemize}

    \item \textbf{Timed automata} are a standard modelling formalism
      for real\hyp{}time systems because they enable efficient system
      verification by model checking and testing.

    \item Basically, a timed automaton is a finite\hyp{}state machine
      extended with clock variables.

    \item The time denoted by the clocks is dense or continuous, in
      other words, it is a mathematical real number.

    \item Moreover, edges of the automata are threefold, in all
      generality:
      \begin{enumerate}

        \item not only do they carry \textbf{send or receive} actions
          for synchronisation on unbuffered channels,

        \item but also \textbf{guards}, which are constraints on the
          clock values that must hold for the edges to be executable,

        \item and \textbf{clock resets} to zero.

      \end{enumerate}
  \end{itemize}
\end{slide}

\begin{slide}
  \title{Timed automata}

  \begin{itemize}

    \item The semantics of a network of timed automata is their
      synchronous product.

    \item This kind of model is usually extended further with bounded
      discrete variables which are part of the system state and are
      used as if in an imperative language.

    \item In sum, the state of the system consists in the locations
      (vertices) of all the automata, the clock values and the
      discrete variable values.

  \end{itemize}
\end{slide}

\begin{slide}
  \title{UPPAAL}

  \begin{itemize}

    \item The model checker \UPPAAL uses an extension of timed
      automata, called \textbf{timed safety automata} because the
      theoretical model is further extended with features which help
      in proving safety and liveness properties, like urgent and
      committed locations, urgent synchronisations and broadcast
      channels.

    \item User functions are allowed for computational tasks and have
      a \textsf{C}~look and feel, except for the absence of
      pointers.

    \item Synchronisations are realised only with rendezvous on one
      action name: the sender is blocked until the receiver is in a
      state with an out\hyp{}going executable edge receiving the same
      action; the sender is released after the receiver has accessed
      and perhaps modified the global state and reached the next local
      state (vertex).

    \item In other words, there is no parameter passing with actions:
      instead, shared variables and side\hyp{}effects are used to
      exchange information (bidirectionally) during the rendezvous.

  \end{itemize}

\end{slide}

\begin{slide}
  \title{Temporal Computational Temporal Logic}

  \begin{itemize}

    \item The query language of the model checker for \UPPAAL is based
      on \textbf{Timed Computational Temporal Logic} (TCTL), a modal
      extension of \textbf{Computational Temporal Logic} (CTL) with
      time.

    \item Temporal logics only express constraints on the order of
      events, their timed extensions can add quantitative constraints
      on delays between those events.

    \item TCTL adds two important concepts to LTL:
      \begin{enumerate}

        \item Branching time: The concept of future in the TCTL is
          that of a tree of all possible traces from the present
          moment. The objects of the logical formulas are paths (in
          the tree) and individual states.

        \item Clock variables and clock constraints, making it
          possible to reason about clocks and their values.

      \end{enumerate}

  \end{itemize}

\end{slide}


\begin{slide}
  \title{TCTL}

  \begin{itemize}

    \item Quantifications are universal or existential: $\textsf{A} \,
      \varphi$, which means that, from the present state, ``for all
      paths, the formula $\varphi$~holds,'' and $\textsf{E}\,
      \varphi$, which means that ``there exists a path such that
      $\varphi$~holds.''  Furthermore, we find the following modal
      operators:
      \begin{itemize}

        \item $\textsf{X} \, \varphi$, meaning: ``In the next state,
          $\varphi$ is true;''

        \item $\textsf{F}\,\varphi$, meaning: ``In a future state,
          $\varphi$ is true;''

        \item $\textsf{G} \, \varphi$, meaning: ``Globally in the
          future, $\varphi$ is true;''

        \item $\varphi \mathrel{\textsf{U}} \psi$, meaning:
          ``$\varphi$ is true \textbf{until} $\psi$ is true.''

      \end{itemize}

    \item For example: $\textsf{AG}(\varphi \rightarrow
      (\textsf{EF}\,\psi))$ means: ``It is globally the case that, if
      $\varphi$ holds, then there exists a path such that, at some
      point in the future, $\psi$ holds as well.''

  \end{itemize}
\end{slide}

\begin{slide}
  \title{Statistical model checking}

  \begin{itemize}

    \item There is an extension of model checking which attracts a lot
      of attention nowadays: \textbf{statistical model checking}.

    \item Simply put, the timed automata are extended with
      probabilities to become \textbf{Priced Timed Automata}, and a
      stochastic semantics is used.

    \item The temporal logic TCTL is also extended conservatively to
      support probabilistic queries.

    \item The main interest of these extension is to have model
      checking go beyond worst case analyses, like the worst case
      response time of a recurrent task following a given schedule,
      and assess the average case behaviour of real\hyp{}time systems.

    \item In other words, statistical model checking decides whether a
      system statisfies a property \textbf{with some degree of
        confidence}.

  \end{itemize}

\end{slide}

\begin{slide}
  \title{SPIN versus UPPAAL}

  \begin{itemize}

    \item Perhaps the most obvious differences between the modelling
      languages of \SPIN and \UPPAAL are the absence of real\hyp{}time
      (clocks) in the former and the absence of asynchronicity
      (buffered channels) in the latter.

    \item \Promela allows several kinds of synchronisations, whilst
      \UPPAAL only offers binary rendezvous (except for broadcast) and
      the sharing of variables. In other words, \Promela features
      synchronisation by message passing, whilst \UPPAAL uses
      rendezvous and shared variables.

    \item In particular, this means that the exchange of information
      in \UPPAAL can be bidirectional.

  \end{itemize}
\end{slide}

\begin{slide}
  \title{SPIN versus UPPAAL}

  \begin{itemize}

    \item As far as data types are concerned, both \Promela and
      \UPPAAL have primitive and structured data types, including
      arrays, although \Promela does not allow arrays to be passed as
      a parameter during synchronisation.

    \item Buffered channels can be modelled explicitly in \UPPAAL as
      timed automata with an array of messages as a local variable,
      but this may quickly overload the state space, as all
      permutations of the messages are observable.

    \item In \Promela, channels are first\hyp{}class objects, that is,
      they are values and they can be sent from one process to another
      in order to create a private communication channel.

    \item By contrast, in \UPPAAL, channels are not a concept and they
      have to be inferred on the basis of the synchronisations, i.e.,
      if two automata can synchronise, an implicit channel exists
      between them.

  \end{itemize}
\end{slide}

\begin{slide}
  \title{SPIN versus UPPAAL (continued)}

  \begin{itemize}

    \item It is possible in \Promela to create at most~255 processes
      at run\hyp{}time, whilst \UPPAAL requires a statically fixed
      number of automata (so\hyp{}called template instanciations).

    \item To a certain extent, \Promela and \UPPAAL can simulate each
      other, even if it is an open question as to what are the largest
      subsets of these languages that are equivalent by bisimulation.

    \item Nevertheless, in practice, the models of \Promela often use
      asynchronous channels and the models of \UPPAAL often rely
      prominently on clocks, which makes the two languages
      incommensurable.

  \end{itemize}
\end{slide}

\begin{slide}
  \title{LTL versus TCTL}

  \begin{itemize}

    \item The timed nature of TCTL makes it trivially incomparable to
      LTL.

    \item The untimed fragment of TCTL, which is CTL, has \textbf{not}
      the same expressive power as LTL.

    \item A CTL formula and a LTL formula are said equivalent if, for
      all models expressed as Labelled Transition Systems, they have
      the same truth value.

    \item It can be shown that there are formulas in CTL which have no
      equivalent in LTL, and vice versa, although there exists
      equivalent formulas.

  \end{itemize}

\end{slide}
\begin{slide}
  \title{Synchronous dataflow languages}

  \begin{itemize}

    \item By contrast, note that in the control system industry (e.g.,
      automotive industry, aerospace, robotics, logic circuit design,
      etc.) \textbf{synchronous languages} like \Esterel, \Lustre,
      \Simulink, etc. are used to express the models.

    \item These enable the automatic generation of the implementation,
      but they assume static functional blocks and a global clock.

    \item By contrast, the application domain of \SPIN is that of
      \textbf{dynamic} asynchronous software systems, which allow the
      instantiation and scheduling of new processes at run\hyp{}time,
      and where processes may perform complex computational tasks in
      addition to protocol related operations.

    \item Blockchain network protocols are waiting to be formally
      modelled, and their critical properties could be verified.

  \end{itemize}

\end{slide}

\begin{slide}
  \title{Model checking in a nutshell}

  \begin{itemize}

    \item The \textbf{model-driven design} of protocols uses formal
      models, possibly including real\hyp{}time features, which are
      checked with regards to some desired properties and tested
      through simulation, before implementations are derived.

    \item \textbf{Model checking} is a formal method whereby the truth
      of a logical formula (e.g., of a temporal nature) is checked on
      a finite model of a system.

    \item The model is a \textbf{non\hyp{}deterministic finite
      automaton} (NDFA), and the claim is the \textbf{negation} of the
      property (as a NDFA too).

    \item The \textbf{product} of these automata is computed: if a
      path in the model is a run in the product, then the property is
      false and a \textbf{counterexample} is the path in the model,
      otherwise the property holds for all paths.

  \end{itemize}
\end{slide}

\begin{slide}
  \title{Model checking compared}

  \begin{itemize}

    \item Model checking is different from \textbf{testing}, which
      consists in interacting with an implementation, whose state
      space may be infinite, and observing whether its responses
      conform to its model (if no stimulation of the system is
      performed, we have an instance of \textbf{passive testing}).

    \item Model checking is also different from \textbf{abstract
      interpretation} because it does not rely on induction when
      interpreting the states of the system: model checking really
      exhausts them one by one.

    \item Testing hence applies to implementations and can, at best,
      only prove the presence of errors, whilst model checking applies
      to models and conclusively proves the absence of design errors
      or finds a counterexample (with an execution trace, called
      \textbf{scenario} in telecommunications), thanks to the
      finiteness imposed by construction.

  \end{itemize}

\end{slide}

\begin{slide}
  \title{Model checking and testing}

  \begin{itemize}

    \item Clearly, model checking and testing are complementary, and
      it even makes sense to ``test'' a model by \textbf{simulation},
      that is, by animating runs of the model itself and thusly derive
      \textbf{test sequences} (of inputs and outputs), ranging from
      purely random walks to guided explorations, for example to cover
      all the transitions of a component.

    \item \textbf{Model\hyp{}based test generation} is usually a
      heuristics which avoids the potential combinatorial explosion of
      the state space at the cost of incompleteness.

  \end{itemize}

\end{slide}

\begin{slide}
  \title{Notions of temporal logic}

  \begin{itemize}

    \item The properties or, put more properly, the claims to be
      checked on the models are expressed in some temporal logic, that
      is, a modal logic interpreted over time.

    \item The most common properties expected from a system are
      \begin{itemize}

        \item \textbf{reachability}: ``Something good may
          happen;''

        \item \textbf{liveness}: ``Something good eventually
          happens;''

        \item \textbf{safety}: ``Something good always happens.''

      \end{itemize}

    \item Reachability is a weak property, which is nonetheless useful
      to debug a model under construction, for example, by asserting
      that a process may access a given resource (a scenario can
      usually be automatically constructed).

  \end{itemize}

\end{slide}

\begin{slide}
  \title{Safety}

  \begin{itemize}

    \item Safety is a strong property reminiscent of a global
      invariant in non\hyp{}temporal logics, and it can be utilised
      for example
      \begin{itemize}

        \item to state \textbf{mutual exclusion}, that is, that, at
          any time, there will be no more than one process accessing a
          common resource,

        \item or the absence of \textbf{deadlocks}, that is, that, at
          any time, there will be no processes waiting transitively on
          each other for the release of a lock on a shared
          resource.

      \end{itemize}
      Other interesting safety properties are the non\hyp{}reception
      of unspecified messages and the absence of invalid end states.

  \end{itemize}

\end{slide}

\begin{slide}
  \title{Safety versus liveness}

  \begin{itemize}

    \item In a temporal logic, doing nothing obviously realises
      safety, and this is where liveness comes into play, to require
      some progress.

    \item For example, liveness is useful for expressing the absence
      of \textbf{starvation}, in other words, the fact that a process
      will gain access to a resource.

    \item In the theory of programming languages, \textbf{partial
      correctness}, by which, if a program terminates, then its output
      satisfies some condition, would be considered a safety property
      in temporal logics, whilst \textbf{termination} would be
      considered a liveness property. (Partial correctness and
      termination entail \textbf{total correctness}.)

    \item A criterion to distinguish between safety and liveness is
      the following: safety can, in theory, be \textbf{disproved} in
      finite time, by simple observation of the states of the system
      execution, whereas liveness can be \textbf{proved} in finite
      time.

  \end{itemize}

\end{slide}

\begin{slide}
  \title{Fairness}

  \begin{itemize}

    \item Let us note that temporal logics assume infinite futures,
      and since the system is finite, its behaviour must then be
      cyclic, as this happens to be the case of many communication
      systems implementing sessions or interactive loops, like servers
      and command shells.

    \item This aspect is perhaps addressed more obviously by
      properties like \textbf{fairness}, which states that ``something
      good will happen an arbitrary number of times,'' and can be
      conceived as a repeated liveness property.

    \item For example: ``If access to the critical section is
      infinitely often requested, then access will be granted
      infinitely often.''

    \item Fairness is most useful when dealing with the scheduling of
      processes.

  \end{itemize}

\end{slide}

\begin{slide}
  \title{Mastering complexity}

  \begin{itemize}

      \item The increase in computing power and decrease in hardware
        costs both contribute to the pertinence of formal methods, and
        even sometimes simply unlock their application to certain
        domains.

      \item What is at stake here, indirectly, is \textbf{mastering
        the size and the logical complexity of computing systems},
        which is the main topic of software engineering.

      \item Formal methods are also of interest to the
        \textbf{ordering party}, that is, the party that commissions,
        in part or total, the development of a software from a third
        party, the \textbf{supplier} (the remaining party being the
        \textbf{customer}).

  \end{itemize}

\end{slide}

\begin{slide}
  \title{The limits of tests}

  \begin{itemize}

    \item  The ordering party must make sure that
      \begin{enumerate}

        \item their specifications and requirements are not ambiguous
          and cover their expectations,

        \item the delivered product matches the specifications.

      \end{enumerate}

    \item Concerning the second point, the ordering party must at
      least validate the product, and, in that aim, they require that
      it passes a \textbf{battery of tests}.

    \item The complexity of the tests increases greatly with that of
      the product and of its new versions (i.e., proportionally to the
      number of system states).

    \item These tests can be manual, but the \textbf{automatic
      generation of test suites} from formal specifications or
      \textbf{simulations} is gaining traction.

  \end{itemize}
\end{slide}

\begin{slide}
  \title{The limits of tests}

  \begin{itemize}

    \item Testing is always \textbf{necessary} but can prove
      \textbf{insufficient}, depending on the level of requirement,
      because tests only cover a finite number of system
      behaviours.

    \item The structural coverage of all the branches in a program
      does not take into account all the possible combinations of the
      input values (which can be infinite in theory).

    \item A failing test proves the presence of a fault (bug), but a
      passing test is at best a non\hyp{}regression test or is simply
      non\hyp{}conclusive because it \textbf{does not prove the
        absence of errors}.

    \item Sometimes the ordering party provides the test suite that
      the product must pass. This is problematic if the supplier
      intentionally builds a product that pass the tests (thus a
      finite number of use cases), but is faulty otherwise.

    %% \item If the product has been developed with the support (even
    %%   partial) of formal methods, then it can be delivered with a
    %%   \textbf{proof of its conformity} (even partial), which the
    %%   ordering party can then check. (It is much easier to verify the
    %%   correctness of a mathematical proof than discovering it.)

  \end{itemize}
\end{slide}

\begin{slide}
  \title{Stakes}

  \begin{itemize}

    \item National bodies and institutions are aware of the progress
      of formal methods. For instance, in the transport, aerospace,
      medical devices and smart cards industries, they require the use
      of formal methods when certain thresholds of safety and security
      must be met.

    \item We can observe both an increase in the number of fields of
      applications (operating systems, robotics, e-commerce, air
      traffic control, etc.) and a rise in the requirements.

    \item In certain cases (transport, control systems in nuclear
      plants, implanted medical devices), human lives are at stake.

    \item Almost since their inception, rising industries, like the
      internet of things and blockchain networks, can now benefit from
      formal methods, instead of waiting many years.

  \end{itemize}
\end{slide}

\begin{slide}
  \title{Stakes}

  \begin{itemize}

    \item There is now a long string of textbook examples of system
      failures due to faulty software, incurring financial losses,
      with or without malicious intent.

    \item Perhaps worth of notice are the recent \textbf{attacks on
      the Ethereum blockchain} network, which targeted erroneous smart
      contracts (programs representing an agreement between two peers
      over the network). We will come back to this topic in another
      lecture.

    \item Currently, formal methods apply to programs up to hundreds
      of thousands lines of code (LOC), or even more, depending on the
      language and the level of accuracy: \textbf{this is not
        negligible.}

  \end{itemize}

\end{slide}

\begin{slide}
  \title{Communication and rigour}

  \begin{itemize}

    \item One overlooked aspect of a rigorous or formal semantics is
      to \textbf{facilitate the communication}, by providing a clear
      framework for the discourse and also an impartial arbiter. It is
      also an excellent guide to build support tools (as in compiler
      construction). It becomes then possible to express some expected
      properties of a system, and even to prove some of them.

    \item Let us keep in mind that the maintenance of software systems
      swallows in average two thirds of the total cost of a project,
      and that correcting an error in a specification requires
      20$\times$ more efforts than if detected during the exploitation
      phase.

    \item Formal methods bring an \textbf{extra rigour} to the
      software industry in order to reduce those costs.

  \end{itemize}

\end{slide}

\begin{slide}
  \title{The method in ``Formal methods''}

  \begin{itemize}

    \item Software engineering shows that it is important to spend a
      significant effort during the first phases of the life cycle, in
      particular, by establishing \textbf{trustworthy specifications},
      that is,

      \begin{itemize}

        \item that match what is intuitively expected,

        \item that are coherent.

      \end{itemize}

    \item The first point relies on a good knowledge of the customers'
      business and needs, a knowledge that the customers themselves
      cannot always express. A Socratic dialogue is needed between
      customer and designer in order to delineate the properties of
      the expected result.

    \item Step by step, the designer builds a \textbf{formal model}
      that is not always shared with the customer, but to whom they
      feed back the logical implications, with the aim of
      \textbf{refining} those properties.

  \end{itemize}
\end{slide}

\begin{slide}
  \title{The method in ``Formal methods''}

  \begin{itemize}

    \item The designer can also opt for a \textbf{fast prototyping},
      which yields in a short lapse a new version of the system, one
      that seems more desirable.

    \item Early concerns about the cleanliness or efficiency of the
      pieces of software can hamper the process at this stage.

    \item When the prototyping phase is over, priorities change and
      the quality goals come to the fore.

    \item The use of \textbf{functional languages}, like \OCaml,
      simplify the transition from prototyping to the product.

  \end{itemize}

\end{slide}

\begin{slide}
  \title{In short}

  Software engineering shows that it is important to spend a
  significant effort during the first phases of the life cycle of a
  project and
  \begin{itemize}

    \item understanding the customer’s business,

    \item establishing trustworthy specifications,

    \item validating and refining with the customer the specification,

    \item creating prototypes of the system.

  \end{itemize}

\end{slide}

\begin{slide}
  \title{Formal languages as Ariadne's threads}

  \begin{itemize}

    \item The first step towards a solution is to formulate the
      problem in a formal or semi-formal language.

    \item That is because a \textbf{formal language} can be given a
      clear semantics, at least less ambiguous than expressed in
      English, and it is founded on proven mathematical theories.

    \item This enables the proof of some desired \textbf{properties of
      the system} at the specification level, on the one hand, and at
      the code level, on the other hand.

    \item It becomes then possible to
      guarantee the \textbf{conformity} of the latter with respect to
      the former:
      \begin{itemize}

        \item the \textbf{correctness} (or \textbf{soundness})
          property states that all the values returned by the program
          are indeed expected by the specification;

        \item the \textbf{completeness} property states that all
          expected values can be computed by the program.

      \end{itemize}
  \end{itemize}
\end{slide}

\begin{slide}
  \title{Methodology}

  In short, the first steps of any formal methodology consists in
  \begin{enumerate}

    \item formulating the problem in a formal or semi\hyp{}formal
      language,

    \item expressing the desired properties at the specification
      level,

    \item proving those properties,

    \item prove the correctness and completeness of the implementation
      with respect to the specification.

  \end{enumerate}

\end{slide}

\begin{slide}
  \title{Properties}

   For reasons stemming from the very foundations of mathematics,
   these properties, as others, may not always be provable in general
   ---~logicians call them \textbf{undecidable}. (We will revisit what
   his means later.)

   \bigskip

   There are many ways to address the issue of conformity:
   \begin{itemize}

     \item Hoare logic (pre- and postconditions),

     \item the enumeration of reachable states (model checking),

     \item the refinement of specifications (B method),

     \item rewrite systems,

     \item arithmetic,

     \item program extraction from a proof (\Coq), etc.

   \end{itemize}

\end{slide}

\begin{slide}
  \title{Tooling}

  A formal language lends itself well to the elaboration of
  \textbf{software tools} to automatically or interactively support
  the different tasks of software engineering.

  \bigskip

  As a consequence, it becomes easier and cheaper
  \begin{itemize}

    \item to program a systems,

    \item to test it,

    \item to maintain it.

  \end{itemize}

  \bigskip

  This yields
  \begin{itemize}

    \item an increased mastering of the life cycle,

    \item a more reliable documentation.

  \end{itemize}

\end{slide}

\begin{slide}
  \title{Formal languages for the ordering party}

  Formal languages also help the ordering party as
  \begin{itemize}

    \item specifications can be reviewed,

    \item products can be checked against their specifications.

  \end{itemize}

  \bigskip

  Other upsides are
  \begin{itemize}

    \item the shortening of the test and integration phases,

    \item the early identification of corner cases.

  \end{itemize}

\end{slide}

\begin{slide}
  \title{Some upsides}

  \begin{itemize}

    \item Formal methods exploit a lot the initial phases of the
      development (specification, design), at the risk of giving the
      negative impression of being too time consuming.

    \item Nevertheless, later phases (test, integration) are shortened
      and better mastered.

    \item Quite early, the formalisation, that is, the use of a logic
      for modelling, brings forth many delicate aspects.

    \item These delicate points are often inherent to the problem at
      hand, which may therefore seem more complex than at first sight:
      it is important to \textbf{understand where the complexity comes
        from} and not always point the finger at the method.

    \item The use of abstract notations is then a simple necessity,
      not an adventive complication.

  \end{itemize}

\end{slide}

\begin{slide}
  \title{Some caution}

  \begin{itemize}

    \item \textbf{Formal methods are not a panacea}: when facing
      complex problems, there is no easy, obvious way, but informed
      compromises.

    \item One must always keep in mind the distance between the formal
      specification and the reality it models. That gap can only be
      bridged by an approach combining peer reviews, reformulations,
      confrontations to conjectures, Socratic dialogues.

    \item A major obstacle to the adoption of formal methods is a
      \textbf{lack in mathematical culture}, in particular not enough
      familiarity with mathematical notations.

    \item In civil engineering and electronics, mathematics are part
      of daily life, but not in software engineering: \textbf{a shift
        in education is needed}.

  \end{itemize}
\end{slide}

\begin{slide}
  \title{A tour of formal methods}

  Formal methods are quite varied and can be classified in families
  featuring at least
  \begin{itemize}

    \item \textbf{a mathematical framework} with a salient theory (for
      example, transition systems, set theory, universal algebras,
      $\lambda$-calculus),

    \item \textbf{software tools} supported by vendors actively
      maintaining them, but also open source projects,

    \item a preferred \textbf{domain of application} (for instance,
      text processing, databases, real-time computing, network
      protocols, parsing),

    \item and \textbf{a community} of knowledgeable and enthusiast
      users, sometimes belonging to different schools of thought, but
      sharing the same interests.

  \end{itemize}
  We can look at formal methods with different angles: whether they
  are \textbf{specialised or general}, or whether they favour
  \textbf{specification or verification}.

\end{slide}

\begin{slide}
  \title{Specialised approaches}

  \begin{itemize}

    \item The specification of a system spans a large set of aspects,
      including architecture, observable behaviours (I/O,
      communication), data representation and storage (internal or
      external), etc.

    \item \textbf{Specialised methods} are going to support well one
      aspect.

    \item Some methods model systems by privileging \textbf{data
      storage and processing}, perhaps from tables or streams. For
      example, a web service could focus on database management.

    \item Other methods will focus on \textbf{information exchange},
      which can be modelled by the sharing of data (in a central
      memory), by synchronous communication (which assumes a unique
      clock without propagation delays) or asynchronous by message
      interchange, or remote procedure calls, etc.

    \item An example would be a switch, which is a telecommunication
      system that routes messages (like landlines telephone calls).

  \end{itemize}

\end{slide}

\begin{slide}
  \title{General approaches}

  \begin{itemize}

    \item Other methods are less constrained and offer an excellent
      general interpretative framework.

    \item By often focusing on the operational aspects, like tooling,
      specialised approaches can obfuscate the global understanding
      and track with difficulty the evolution of the subject being
      studied.

    \item What \textbf{general approaches} offer, by lying
      \textbf{closer to mathematical logic}, is a greater freedom of
      expression, but they lack in methodology.

  \end{itemize}
\end{slide}

\begin{slide}
  \title{State for the specialised approaches}

  \begin{itemize}
    \item A typical example contrasting these two different approaches
      is the role they give to the \textbf{state of a system}.

    \item As we saw earlier, the state is all the information
      characterising a system running at a given point in time (memory
      contents, program counter, files, network buffers, etc.)

    \item State is obvious when we consider a \textbf{database}: its
      contents makes up for a greal deal of the overall system
      information.

    \item When considering a switching telephone network, the state is
      more about the \textbf{global signalling information} needed to
      route the messages, rather than the messages themselves (the
      human voice can be highly compressed).

    \item Also, state seems fundamental if one considers the
      \textbf{devices for executing programs} (computers or virtual
      machines), which imply a memory whose contents may change and,
      perhaps, also a clock.

\end{itemize}

\end{slide}

\begin{slide}
  \title{State for the general approaches}

  \begin{itemize}

    \item On the other hand, state is not fundamental in mathematics.

    \item This is because mathematics focus more on \textbf{composing
      abstractions}, like functions.

    \item For a mathematical function, the state consists of the
      arguments and the constants within the body.

    \item Mathematics can speak about states (think of sets in set
      theory), but it is more difficult to describe the composition of
      instructions affecting a memory, rather than compose purely
      mathematical functions.

    \item In the B~method, there is a notion of \textbf{simultaneous
      assignment} whose interest, by grouping a series of
      non-interfering assignments, consists in \textbf{reducing the
        number of states} a system can reach.

    %% \item Another example is that of \textbf{impure functional
    %%   languages}, like \OCaml, which feature a stateless semantic
    %%   but enable stateful computations: the programmers is trusted
    %%   to handle the state carefully (by means of the so-called
    %%   \textbf{imperative features}, like loops and arrays) and only
    %%   marginally, for optimisation purposes, communication with the
    %%   physical world or with another logical thread.

  \end{itemize}

\end{slide}

\begin{slide}
  \title{Favouring specification or verification}

  \begin{itemize}

    \item We said that another viewpoint for formal methods is whether
      they ease specification or verification.

    \item Indeed, a formal method is made up of two main ingredients,
      \textbf{a specification language} and \textbf{a verification
        system}.

    \item Proof assistants \`a la Boyer-Moore favour the automating of
      proofs at the expense of the ease of expression. At the opposite
      side of the gamut, the design of the formal notation \textsf{Z}
      aimed primarily at providing a large expressiveness and was hard
      to support for tools.

    \item More recent approaches, like \textsf{HOL} or \textsf{Coq},
      try to bring together the best of two worlds: by relying on
      powerful logics to express the application domain, they also
      come along with tools which, on the one hand, help at building
      proofs and, on the other hand, enable the complete verification
      of these with great confidence.

  \end{itemize}

\end{slide}

\begin{slide}
  \title{Agile development and formal methods}

  \begin{itemize}

    \item Agile development is a set of guidelines for software
      engineering that departs from the classic waterfall life cycle.

    \item In particular, the agile approach consistently warns against
      spending too much time ``up front'' on specifications, and,
      instead, includes the customer in an interactive and iterative
      process of feedback during all the development (not just the
      specification phase).

    \item Agile methods are common in many software industries, and
      are not even limited to the software industry.

    \item It is unclear yet if they can be blended with formal
      methods.

  \end{itemize}

\end{slide}

\begin{slide}
  \title{A short review of mathematical logic}

  Mathematical logic grew along different axes:
  \begin{itemize}

    \item model theory,

    \item proof theory,

    \item set theory,

    \item computability theory.

  \end{itemize}
\end{slide}

\begin{slide}
  \title{Model theory}

  There are essentially two manners to specify a system, which are
  complementary:
  \begin{enumerate}

    \item to represent a system by its properties,

    \item or to propose a constructive model of it.

  \end{enumerate}
  Sometimes this duality is expressed as \textbf{specification by
    properties} or \textbf{specification by models}.

  \medskip

  Properties are expressed as \textbf{logical axioms} and the models
  as \textbf{operations on sets}:
  \begin{enumerate}

    \item axioms are the \textbf{syntactic aspect},

    \item operations on sets are the \textbf{semantic aspect}
      (describing the universe of discourse itself).

  \end{enumerate}
  For example, some models satisfying the statement $\forall x.\exists
  y.(y \, R \, x)$ are $(\mathbb{N},>)$, or $(\mathbb{R},<)$, or
  $(\mathbb{N},\mbox{`is a multiple of'})$, etc.
\end{slide}

\begin{slide}
  \title{Truth in model theory}

  \begin{itemize}

    \item The \textbf{semantic consequence} is fundamental in model
      theory and it corresponds to the notion of \textbf{truth}.

    \item A statement~$E$ is a semantic consequence of the
      statements~$A$, $B$, $C$, etc. if \textbf{any} model satisfying
      the properties~$A$, $B$, $C$, etc. also satisfies the
      property~$E$.

    \item We write $A, B, C \ldots \models E$.

    \item This reads: ``$E$ is \textbf{true} if $A$, $B$, $C$,
      etc. are.''

  \end{itemize}

\end{slide}

\begin{slide}
  \title{Proof theory}

  \begin{itemize}

    \item The relationship of semantic consequence is inconvenient
      because it has to be verified on all possible models and there
      may be an infinity of them in general.

    \item We may prefer a provability relationship, or
      \textbf{syntactic consequence}: a statement $E$ is said to be
      \textbf{provable} from the statements~$A$, $B$, $C$, etc. if a
      formal proof of~$E$ can be constructed only from the
      hypotheses~$A$, $B$, $C$, etc. as well as from the axioms and
      the deduction rules of the logic.

    \item (Such a proof is called a \textbf{derivation} in some
      contexts.)

    \item We write $A, B, \ldots \vdash E$.

    \item This reads: ``$E$ is provable if $A$, $B$, etc. $E$ are.''

    \item The statement~$E$ is said to be \textbf{refutable} if its
      negation is provable.

  \end{itemize}

\end{slide}

\begin{slide}
  \title{Soundness and completeness}

  \begin{itemize}

    \item Nevertheless, each step in a proof of a syntactic
      consequence must preserve the meaning (that is, the semantics)
      of the statements: this is called \textbf{soundness}.

    \item Conversely, if all semantic consequence is provable, then
      the theory is said to be \textbf{complete}.

    \item A theory is sound and complete if all provable statements
      are true and vice-versa.

    \item Unfortunately, \textbf{in general, not all true statements
      are provable} (incompleteness theorems of G\"odel).

    \item This is the case already with arithmetics with quantifiers.

    \item Consequently, when we annotate a program with properties, we
      must ask ourselves whether these can actually be proved within
      the chosen logical framework.

  \end{itemize}

\end{slide}

\begin{slide}
  \title{Soundness and completeness}

  In short, now that we have two notions of consequence, we want to
  know how they compare:
  \begin{itemize}

    \item a theory is inconsistent if false statements are provable;

    \item a theory is sound if provable statements are true;

    \item a theory is complete if all true statements are provable;

    \item a theory is sound and complete if all provable statements
      are true and reciprocally.

  \end{itemize}
  While all the theories we use are sound, most of them are not
  complete, meaning there are true statements we cannot prove (the
  theorems of G\"odel explicit under what conditions a theory is not
  complete).

  \medskip

  This is a fundamental, intrinsic, \textbf{limit} to what formal
  logic can do for us.

\end{slide}

\begin{slide}
  \title{Proof theory (resumed)}

  \begin{itemize}

    \item Independently of the connection between the notions of
      syntactic and semantic consequence (that is, provability and
      truth), there are questions that are inherent to the latter,
      like: ``If~$E$ is provable, is there a proof of~$E$ only made up
      of sub-formulas of~$E$?''

    \item If so, the space in which to seek the proof is considerably
      reduced, which empowers an automated proof assistant.

    \item The study of axioms and rules as formal systems (that is,
      how formulas imply other formulas without interpretation), as
      well as their relationship to the notion of semantic consequence
      (that is, how truth begets other truths) is the foundation of
      proof theory.

  \end{itemize}

\end{slide}

\begin{slide}
  \title{Cantor's set theory}

  \begin{itemize}

    \item It is useful to form a set whose elements possess a given
      property: such a definition is said to be a \textbf{definition
        by comprehension} (or \textbf{intension}).

    \item If that kind of definition is left unconstrained, we
      position ourselves in the framework of the first set theory,
      called nowadays na\"{\i}ve, which was found to be inconsistent
      at the start of the twentieth century.

    \item A \textbf{contradiction} is a pair of statements of the
      form~$P$ and~$\neg P$.

    \item A set of statements is \textbf{inconsistent} if a
      contradiction can be derived from them by means of the axioms
      and rules of the logic.

    \item An inconsistent theory can prove anything, and is thus
      worthless.

  \end{itemize}

\end{slide}

\begin{slide}
  \title{Russell's paradox}

  Let us consider Russell's paradox.
  \begin{itemize}

    \item Let us define by comprehension the curious set $R = \{x \mid
      \neg(x \in x)\}$.

    \item If $R \in R$, then~$R$ must obey the property of its
      elements, that is, $\neg(R \in R)$.

    \item Conversely, if $\neg(R \in R)$, then~$R$ satisfies the
      characteristic property of the elements of~$R$, which means that
      $R \in R$.

    \item Therefore, by defining the property $P = R \in R$, we proved
      \(P \Leftrightarrow \neg P\), which is a contradiction.

    \item What this contradiction entails is that
      \begin{quote}
        \textit{the whole question whether a class [set] is or is not
          a member of itself is nonsense.  (Bertrand Russell).}
      \end{quote}

  \end{itemize}
\end{slide}

\begin{slide}
  \title{The axiomatic set theory of Zermelo and Fraenkel}

  \begin{itemize}

    \item A way to avoid the reef of paradoxes consists in
      reformulating the set theory in an axiomatic manner (that is,
      with logical rules) that enforces a \textbf{stratification of
        the constructs}.

    \item Consequently, a property cannot be defined by relying on an
      arbitrary property, as was the case with Russell's paradox,
      where an object was ``defined'' by referencing itself with a
      negation. (\textbf{Impredicativity with negation.})

    \item A modern set theory was proposed by Zermelo and Fraenkel,
      and it yielded a refoundation of mathematics that has been
      devoid of paradoxes up today.

    \item The specification languages~\textsf{Z} and~\textsf{B} are
      built upon that modern theory, plus a zest of type theory to
      bring it closer to the programming language \Ada.

  \end{itemize}

\end{slide}

\begin{slide}
  \title{Type theory}

  \begin{itemize}

    \item Another way to avoid the paradoxes has been proposed by
      Russell himself: \textbf{type theory}.

    \item Type theory also stratifies sets and properties.

    \item At the bottom layer, layer~$0$, lie the individuals.

    \item At layer~$1$ are the sets of individuals (and the properties
      about individuals).

    \item At layer~$2$ are the sets of sets of individuals, etc.

    \item \textbf{Types denote those layers.}

    \item They are associated to the individuals and basic properties,
      and they constrain the construction of the upper layers in such
      way that they can only depend on constructs of the lower layers
      and the \textbf{induction principle}.

  \end{itemize}
\end{slide}

\begin{slide}
  \title{Type theory in programming}

  \begin{itemize}

    \item Type theory has been very fruitful with the programming of
      computers, as many programming languages in recent years feature
      \textbf{type systems}.

    \item Those type systems are logical theories derived from the
      general type theory.

    \item Compilers for these languages use these type systems to
      \textbf{check} type annotations or to \textbf{infer} types, as
      is the case with \OCaml.

    \item The strict layering inherent to type theories is a challenge
      to the definition of \textbf{recursive functions} (which are
      impredicative definitions), so some rules have to be relaxed.

    \item The unpleasant consequence is that some non\hyp{}terminating
      programs may then be given a type, and cannot be rejected by
      compilers.

  \end{itemize}

\end{slide}

\begin{slide}
  \title{Computability theory}

  During the 1940s, several approaches were undertaken to bring the
  concept of computability into the purview of mathematics:
  \begin{itemize}

    \item Turing machines,

    \item $\lambda$-calculus (of Alonzo Church),

    \item partial recursive functions (G\"odel, Herbrand).

  \end{itemize}
  Each theory constitutes a framework in which can be expressed a
  notion of \textbf{effective procedure}, also known as
  \textbf{algorithm}.

  \medskip

  The thesis of Church is that these different formalisms are, in
  fact, equivalent, in the informal sense that they can compute what
  the set of mathematical functions can.

\end{slide}

\begin{slide}
  \title{Countable sets}

  Are all functions computable? To seize that question, a detour
  through the theory of formal languages is needed.
  \begin{itemize}

    \item A set is said to be \textbf{countable} if it is finite or a
      bijection (a one-to-one correspondence) exists between its
      elements and the set of the natural numbers~$\mathbb{N}$ (in
      which case it is infinite). The cardinal of a infinite countable
      set is noted~$\aleph_0$.

    \item The set of the even numbers is countable (the trivial
      bijection is $2n \mapsto n$). In general, any infinite subset of
      the natural numbers is countable.

    \item The set of finite words over the alphabet $\{a,b\}$ is
      countable (the bijection is made up by sorting all the words as
      if entries of a dictionary).

    \item The rational numbers are countable.

    \item Regular expressions are countable.

  \end{itemize}
\end{slide}

\begin{slide}
  \title{Cantor's diagonalisation}

  \begin{itemize}

    \item We may wonder then whether there exists sets whose cardinal
      is strictly greater than~$\aleph_0$, that is, sets that are
      uncountable.

    \item The answer is yes.

    \item For example, \textbf{the set of the subsets of an infinite
      countable set is not countable.}

    \item To see why, we use a technique called
      \textbf{diagonalisation}, introduced for the first time by
      Cantor.

    \item Let a countable set be $A = \{a_0, a_1, \ldots\}$ and
      let~$S$ be the set of the subsets of~$A$.

    \item Let us reason by \textbf{reduction to absurdity} and assume
      that~$S$ is countable. Therefore we can write $S = \{s_0, s_1,
      \ldots\}$.
  \end{itemize}

\end{slide}

\begin{slide}
  \title{Cantor's diagonalisation}

  Let us construct the following infinite table:

  \bigskip

  \begin{equation*}
    \begin{array}{c|cccccc}
      & a_0    & a_1    & a_2    & a_3   & a_4      & \ldots\\
      \hline
      s_0    & \times & \times &        & \times \\
      s_1    & \times & \circ  &        & \times \\
      s_2    &        & \times & \times &        & \times\\
      s_3    & \times &        & \times & \circ  & \\
      s_4    &        & \times &        & \times & \circ\\
      \vdots &        &        &        &        &         & \ddots
    \end{array}
  \end{equation*}

  \bigskip

  It shows which elements of~$A$ belong to which elements of~$S$: a
  cross at the column~$i$ and row~$j$ means that $a_i \in s_j$.

\end{slide}

\begin{slide}
  \title{Cantor's diagonalisation}

  \begin{itemize}

    \item Let us now consider the set \(D = \{a_i \mid \neg(a_i \in
      s_i)\}\).

    \item Its elements are marked with a circle in our table above
      (completing the diagonal): we have $D \subset A$, but is~$D$
      found amongst the~$s_j$?

    \item Let us assume that it is, so there is a~$k$ such that $D =
      s_k$.

    \item We have $a_k \in D \Leftrightarrow \neg(a_k \in s_k)
      \Leftrightarrow \neg(a_k \in D)$, which is a contradiction,
      therefore $D$~is not listed, even though $D \subset A$.

    \item This means that the set of the subsets of an infinite
      countable set is uncountable.

    \item We note~$2^{\aleph_0}$ the cardinal of an uncountable set,
      by analogy with the fact that the cardinal of the set of the
      subsets of a finite set of cardinal~$n$ is~$2^n$.

  \end{itemize}

\end{slide}

\begin{slide}
  \title{Undecidable problems}

  \begin{itemize}

    \item The set of all formal languages is the set of the subsets of
      a countable set (the set of all words), therefore, the previous
      theorem implies that it is \textbf{not} countable.

    \item The solution to a problem corresponds to such a language,
      thus the set of solutions (or problems) is in bijection with the
      set of languages and, thus, is \textbf{not} countable.

    \item An algorithm is a finite chain of symbols, like a word,
      therefore the set of all algorithms is countable.

    \item Consequently, there are more solutions than computable
      solutions (one for each algorithm).

    \item Problems whose solutions are computable are called
      \textbf{decidable problems}, and \textbf{undecidable problems}
      the others.

  \end{itemize}

\end{slide}

\begin{slide}
  \title{Turing-completeness}

  \begin{itemize}

    \item A programming language featuring a conditional construct and
      recursion (or only conditional branching) is enough to express
      all computable functions: this property is called
      \textbf{Turing-completeness}.

    \item This property is very useful but it entails the existence of
      programs that do not terminate for some of their inputs, and
      also the nonexistence of programs that can distinguish them from
      terminating programs (\textbf{G\"odel incompleteness}).

    \item Let us reason by reduction to absurdity and assume the
      existence of a predicate for termination. Let~\(f\) \, be the
      function such that, for any function~\(g\), if~\(g\) \,
      terminates (on all its domain), then \(f \, (g)\) does
      \textbf{not} terminate (has no value), otherwise \(f \; (g)\)
      terminates (has a value).

    \item Then $f\,(f\,)$ does not terminate if~$f$ terminates on
      \textbf{all} its inputs, which is a contradiction, since
      \(f\)~is an input to~\(f\). Thus such a predicate does not
      exist.
  \end{itemize}

\end{slide}

\begin{slide}
  \title{Undecidable problems (redux)}

  \begin{itemize}

    \item This means that the \textbf{termination problem} (or halting
      problem, in the context of Turing machines) is undecidable.

    \item The negation of an undecidable problem is undecidable as
      well.

    \item It is important to know some of these problems because they
      do not have a solution in general.

    \item For \textbf{any} program of a Turing-complete language, the
      following problems are undecidable in general:
      \begin{itemize}

        \item whether it terminates on all its inputs (But a
          \textbf{given} program can be proved to terminate!);

        \item knowing the value of a variable at a given moment of the
          execution;

        \item whether a function call is never performed (\textbf{dead
          code detection}).

      \end{itemize}
  \end{itemize}

\end{slide}

\begin{slide}
  \title{A shift in education}

  \begin{itemize}

    \item The inventors of perpetual motions make us smile nowadays
      because their devices contradict our knowledge of the laws of
      physics: we immediately know that their claims are bogus without
      having to examine them in detail.

    \item Some limits to the application of formal methods can be
      removed by technical progress, but others are inherent to the
      mathematics of what is computable. We have seen that certain
      computing problems are not solvable in general: they are
      undecidable.

    \item The knowledge of intrinsic (theoretical) and extrinsic
      (practical) limits lies within the domain of formal methods.

    \item \textbf{A shift in education is needed} towards more
      mathematics in software engineering.

\end{itemize}

\end{slide}

\end{document}
