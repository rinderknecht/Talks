%%-*-latex-*-

\documentclass{article}

\pagestyle{empty}
\usepackage[T1]{fontenc}
\usepackage{pst-tree}
\usepackage{amsmath}

../TeX/commands.tex

\begin{document}

\pstree[nodesep=2pt,levelsep=17pt,treesep=14pt]{\TR{\fun{sum}}}{
  \pstree{\TR{\texttt{|}}}{
    \TR{\rnode[rt]{n1}{\(n\)}}
    \TR{\texttt{[]}}
  }
}
\hspace*{-4pt}\raisebox{-3pt}{\(\xrightarrow{\alpha}\)}\hspace*{4pt}
\pstree[nodesep=0pt,levelsep=17pt,treesep=14pt]{\TR{\rnode[cb]{Nup}{\(\circ\)}}}{}
\qquad\qquad
\pstree[nodesep=2pt,levelsep=17pt,treesep=14pt]{\TR{\fun{sum}}}{
  \pstree{\TR{\texttt{|}}}{
    \TR{\rnode[rt]{n2}{\(n\)}}
    \TR{\rnode[r]{s}{\(s\)}}
  }
}
\hspace*{0pt}\raisebox{-3pt}{\(\xrightarrow{\beta}\)}\hspace*{-4pt}
\pstree[nodesep=2pt,levelsep=15pt,treesep=14pt]%
  {\TR{\rnode[lb]{plus}{\(+\)}}}{
  \Tn
  \pstree{\TR{\rnode[bc]{sum}{\fun{sum}}}}{
    \Tn
  }
}
\nccurve[angleA=-90,angleB=30,nodesepB=2pt,nodesepA=-1pt]{->}{Nup}{n1}
\ncarc[arcangle=12,nodesep=2pt]{->}{plus}{n2}
\nccurve[angleA=-90,angleB=0,nodesep=2pt]{->}{sum}{s}

\end{document}
