\begin{frame}
  \frametitle{Emails as trees (continued)}

  \begin{itemize}

    \item Consider, for instance, an email message.

    \item It contains at least the sender's address, a subject or
      title, the recipient's address, and a body of text, like:
      \medskip
\begin{center}
\fbox{%
\begin{minipage}{0.8\linewidth}
\begin{alltt}
From: Me\\
Subject: Homework\\
To: You\\

A \textbf{deadline} is a due date for a \textbf{homework}.
\end{alltt}
\end{minipage}
}
\end{center}

  \end{itemize}

\end{frame}

\begin{frame}
  \frametitle{Emails as trees (continued)}

  \begin{itemize}

    \item These elements correspond to nodes in a tree:
      \begin{figure}
        \centering
        \includegraphics{mail}
      \end{figure}

    \item The topmost node (``email'') is the root and the framed
      pieces of text are leaves.

    \item Note that, for historical reasons, computer scientists grow
      their trees upside down, with the root at the top.

    \item The inner (non-leaf) nodes hold ``metadata'', or ``markup'',
      that is, information about the nature of the data contained in
      the subtree.

  \end{itemize}

\end{frame}
