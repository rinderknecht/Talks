

\begin{frame}
  \frametitle{Exceptions}

  Exceptions are values of the predefined type \textbf{exn}.
  \begin{itemize}

    \item The type \textsf{exn} is an \textbf{open variant type}. For
      example:
      \begin{alltt}
        \textbf{exception} Exit \textbf{of} int
      \end{alltt}

    \item \textbf{Raising an exception} is done by the expression
      (\textsf{raise} $e$). The expression~\(e\) must evaluate in an
      exception, which becomes the value of the raise, e.g.,
      \texttt{raise (Exit 2)}.

    \item An exception can be matched if the expression~\(e\) where it
      was raised is embedded in an expression
      \begin{center}
        \Xtry{} $e$ \Xwith{} $p_1 \rightarrow e_1 \mid \ldots
        \mid p_n \rightarrow e_n$
      \end{center}
      \begin{itemize}

        \item If the evaluation of~\(e\) results in a normal value,
          that value is the value of the \textbf{try} expression;

        \item otherwise, the exceptional value is matched against the
          patterns~\(p_i\);
          \begin{itemize}

            \item if the pattern~\(p_i\) matches the exception, then
              \(e_i\)~is evaluated and its value becomes the value of
              the \textbf{try} expression,

            \item else the exception becomes the value of the
              \textbf{try} expression.

          \end{itemize}

        %% \item Note that the patterns for exceptions need not be
        %%   complete, whence a source of potential problems because the
        %%   programmer has to keep track of the set of exceptions
        %%   potentially raised by a given expression. (Contrary to
        %%   \Java, exceptions in \OCaml are no part of the type of
        %%   functions, because this would be deemed cumbersome in the
        %%   presence of higher\hyp{}order functions, like
        %%   \textsf{List.map}.)

      \end{itemize}

  \end{itemize}

\end{frame}

