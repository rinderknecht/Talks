%%-*-latex-*-

\NeedsTeXFormat{LaTeX2e}[1996/12/01]

\documentclass[english]{seminar}

\usepackage[english]{babel}
\usepackage[T1]{fontenc}
\usepackage[latin1]{inputenc}
\usepackage{graphicx}
\usepackage{xspace}
\usepackage{url}
\usepackage{mathpartir}
\usepackage{amssymb,amsmath}
\usepackage{ae,aecompl}
\usepackage{color}

\input{cpp}
\input{slides_look}
\input{asn1}

\newtheorem{proposition}{Proposition}
\newtheorem{theorem}{Theorem}
\newcommand{\id}[1]{\textsf{#1}}
\newcommand{\coding}[4]{#1 \vdash #2 : #3 \rightarrow #4}
\newcommand{\decoding}[3]{{\cal D}(#1,#2,#3)}

%---------------------------------------------------------------------
% Title
%
\begin{document}

\division{debut}{\url{christian_rinderknecht@yahoo.fr}}%
         {Konkuk University, Seoul, October 18th, 2004}

\begin{slide}
\grandtitre{Illustrated professional overview}

\bigskip

\centerline{Christian Rinderknecht}

\end{slide}

\centerslidesfalse

%---------------------------------------------------------------------
% In short
%
\begin{slide}
\titre{In short}

\begin{itemize}

  \item Ph.D. at INRIA (Rocquencourt) in the Cristal team
  (\url{http://cristal.inria.fr}) from 1993 to 1997. Application of
  formal techniques to \ASN, a specification language used in
  telecommunications (design and development of a compiler front-end
  and soundness proof of an encoding rule). \myblue{Theoretical
  computer science background.}

  \item Alternating academic and industrial positions, with common
  point the modelisation of technical problems, conception of
  solutions and software development. From Sep. 2001 to Sep. 2002,
  researcher at \emph{Information and Communications University}
  (Daejeon).

  \item From 2003 until Sep. 2004, assistant professor at a French
  private engineering school (\url{http://www.devinci.fr}).

%  \item Wish to settle down in Korea.

\end{itemize}

\end{slide}

%---------------------------------------------------------------------
% Research topics
%
\begin{slide}
\titre{Research topics}

\begin{itemize}

  \item Protocol engineering in telecoms (INT, ICU, de Vinci):
  specification analysis, formal design review, methods and software
  tools for automated test suite generation. ($\approx$ four years)

  \item Software development cycle (Alcatel, PolySpace): compilation,
  static program analysis, tests, code quality ($\approx$ two years).

  \item At \emph{L�onard de Vinci}, XML-based web-site frameworks
  supporting e-learning, but not exclusively. ($\approx$ one year)

  \item In all cases, my work was to lead and apply some research, the
  approach was rigourous (sometimes formal), often guided by the
  \myblue{``language'' aspect of the problems}.

\end{itemize}

\end{slide}

%---------------------------------------------------------------------
% Projects and collaborations
%
\begin{slide}
\titre{Projects and collaborations}

\begin{itemize}

  \item During my Ph.D. I collaborated closely with France Telecom
  R\&D. Since then I have been in contact with the \ASN Project leader
  at the International Telecommunication Union (ITU).

  \item At the French National Institute of Telecommunications (INT) I
  took part to two national projets, involving both academic and
  industrial partners.

  \item At \emph{L�onard de Vinci}, I was a member of the European
  project \textsc{Kaleidoscope} (e-learning, XML frameworks, semantic
  web, ontologies etc.) and of the AIDA seminar
  (\emph{Interdisciplinary Approach for Interactive Learning
  Environments}).

\end{itemize}

\end{slide}


%---------------------------------------------------------------------
% Teaching
%
\begin{slide}
\titre{Teaching}

\begin{itemize}

  \item During my Ph.D. (1993-1996), I teached at univ. Ren� Diderot
  (Paris VII) to undergraduates: algorithmics, Pascal, \cpp, linear
  algebra, integral calculus. I designed and tutored student projects,
  prepared exams etc.

  \item Teaching assistant at \emph{Institut Sup�rieur d'�lectronique
  de Paris}, an engineering school: \cpp{} programming (1993-1996).

  \item Teaching assistant in algorithmics and functional programming
  with Caml (1996-1997).

\end{itemize}

\end{slide}

%---------------------------------------------------------------------
% Teaching (bis)
%
\begin{slide}
\titre{Teaching (continued)}

\begin{itemize}

  \item Professional courses: Pascal (1993) and \cpp{} (1999).

  \item At INT, an engineering school: supervision of last-year
  projects and tutoring of a master student; teaching of
  \emph{Semantics Definition Language} (SDL) to students and
  professionals (in Spanish), from 1998 to 2000.

  \item At \emph{L�onard de Vinci}, I taught software engineering,
  introduction to formal methods, Java programming, Objective Caml
  (functional programming), Unix shell, \textsc{gnu} development
  tools, compiler generation, introduction to protocol engineering. I
  supervised student's projects and training periods in companies.

\end{itemize}

\end{slide}

%---------------------------------------------------------------------
% ICU
%
\begin{slide}
\titre{Visiting Professor at Information and Communications University (2001-2002)}

\vspace*{-5pt}

\begin{itemize}

  \item Full-time research position. I published a journal paper about
  \ASN. I learnt new issues in mobile networks, like \emph{ad hoc}
  networking.

  \item \emph{Abstract Syntax Notation One} (\ASN) is a standard
  language for definition of types whose values may be exchanged at
  run-time by communicating applications, independently of the
  possible heterogeneity of the peers. Applications: network equipment
  management (SNMP), e-mail (X.400), mobile telephony, real-time
  multimedia communication over the internet (MHEG, H.225, H.245,
  H.323), fax over the internet (T.38), cryptographic protocols (SET,
  OpenSSL, PKCS~7, Kerberos, biometrics) and databases (X.500
  directory, Z39.50 (U.S. Library of Congress), DNA repositories).

  \item I solved formally the important problem of validating \ASN
  specifications by borrowing technical concepts from another field
  (set constraints).

\end{itemize}

\end{slide}


%---------------------------------------------------------------------
% L�onard de Vinci
%
\begin{slide}
\titre{Assistant Professor at \emph{L�onard de Vinci} (2003-2004)}

\begin{itemize}

  \item I published two papers on two very different topics: 

  \begin{itemize}

    \bitem a question-answering mining web-platform,

    \bitem the formal design review of \ASN and BER.

  \end{itemize}

  \item \ASN is a high-level data (\emph{versus} control) notation for
  protocols

  \item consisting of data and type definitions

  \item and network concepts modeling features (like PDUs),

  \item embedded in other specification languages like SDL (control),
  MSC (functional requirements), TTCN (test suites) etc.

  \item also used without formal control specification by protocol
  designers and tool implementors, e.g. SNMP, OpenSSL.

\end{itemize}

\end{slide}

% ------------------------------------------------------------------------
% What are the Basic Encoding Rules (BER)?
%
\begin{slide}
\titre{What are the Basic Encoding Rules (BER)?}

\begin{itemize}

  \item The BER are a set of rules defining the serialization into
  bits of \ASN values;

  \item the bit frames (\emph{codes} or \emph{encodings}) have a
  uniform structure: a triple made of a \emph{tag}, a \emph{length}
  and some \emph{contents};

  \item the tag field is extracted from the data type,

  \item the contents field is extracted from the data itself,

  \item the length field is computed over the contents field,

  \item the structure allows embedding of sub-frames (in the contents
  field)

  \item and an \emph{indefinite length} form, as a sender's option.

\end{itemize}

\end{slide}

% ------------------------------------------------------------------------
% How do ASN.1 and the BER interact?
%
\begin{slide}
\titre{How do \ASN and the BER interact?}

\begin{itemize}

  \item Both peers share a common \ASN specification and agree upon
  the usage of the BER;

  \item each peer compile the \ASN specification to a set of type
  declarations and a set of coders/decoders (codecs) for each type;

  \item the target language may be different depending on the peer;

  \item the coders transform a value into a stream of bits according
  to the BER,

  \item the decoders transform a stream of bits into a value according
  to the BER (or reject it as ill-formed);

  \item these pieces of source code are compiled and linked separately
  against the corresponding communicating applications.

\end{itemize}

\end{slide}

% ------------------------------------------------------------------------
% So, what is wrong?
%
\begin{slide}
\titre{So, what is wrong?}

\begin{itemize}

  \item In the last few years, the press has reported several alleged
  vulnerabilities

  \item relating \ASN and the BER to SNMP and OpenSSL

  \item about improper decoding (and rejection) of ill-formed BER
  encodings

  \item leading to buffer overflows,

  \item unspecified (non-deterministic) behaviours,

  \item stack corruptions,

  \item hence possible denials of service.

  \item Further study proved that all the problems lay actually in
  \emph{implementations}:

  \item \myblue{we want to show that \ASN and the BER are, themselves,
  well designed.}

\end{itemize}

\end{slide}

% ------------------------------------------------------------------------
% Modeling
\begin{slide}
\titre{Modeling (assumptions and encoding)}

\vspace*{-6pt}

We start making some assumptions that abstract away low-level details
and help us capturing the design principles of the BER:

\begin{itemize}

  \item values are given in \ASN, not in the application language (in
  fact they are produced in memory at run-time);

  \item codes are tuples, not bit series, in order to easily reason by
  induction on their structure;

  \item the network is reliable (no alteration of codes).

\end{itemize}

BER codes may not be unique for a given value, due to possible
sender's options (e.g. encoding of booleans or unordered sets).

\begin{proposition}
All the BER encodings of a given value, according to a given type, are
equivalent.
\end{proposition}

\end{slide}

% ------------------------------------------------------------------------
% BER decoding
%
\begin{slide}
\titre{Modeling (decoding)}

\begin{itemize}

  \item The BER standard (26~pages of English) does not define the
  decoding at all.

  \item We believe that the receiver is the peer most concerned with
  the sound design of the BER: \myblue{the composition of encoding and
  decoding yields a value equivalent to the original one}.

  \begin{theorem}[Soundness]
    Let $v$ be a value of type \T. Then the BER decoding of \emph{any}
    BER encoding of $v$ is equivalent to $v$.
  \end{theorem}

  Hence we need an equivalence relationship between values
  (thus independent from the BER).

\end{itemize}

\end{slide}

% ------------------------------------------------------------------------
% Soundness property
%
\begin{slide}
\titre{Soundness property}

\centerline{\includegraphics[width=.85 \textwidth]{model-1.eps}}

\end{slide}

% ------------------------------------------------------------------------
% Equivalence entailment
%
\begin{slide}
\titre{Equivalence entailment}

As a consequence, in order to prove the soundness property, it appears
that we need to link logically the two equivalence relationships we
defined separately on codes and values.

\begin{proposition}[Equivalence entailment]
Let $c_1$ and $c_2$ be two equivalent codes. Then the decoding of
$c_1$ is equivalent to the decoding of $c_2$ assuming the same type.
\end{proposition}

One equivalence rely solely on the BER and the other one is inherent
to \ASN. Hence \myblue{this proposition is the link between the design
of \ASN and the design of the BER}.

\end{slide}

% ------------------------------------------------------------------------
% Core ASN.1
% 
\begin{slide}
\titre{Core \ASN}

\vspace*{-4pt}

\begin{itemize}

  \item The BER only apply to a subset \ASN (whose definition spans 
  188~pages).

  \item This suggests that the whole \ASN can be reduced to an inner
  subset which has the same expressivity, we call \myblue{BER domain}.

  \item In fact it is even possible to reduce further the BER domain
  to a smaller subset, we call \myblue{\core}, in which formal
  reasoning is easier.

  \item The formal definition of these reductions is achieved by means
  of a series of rewriting rules that preserve the set of values of
  the specifications (or of a given type).

  \item We can easily check some useful properties in \core and
  enforce e.g.

  \begin{theorem}[BER termination]
    The encoding of \core values with the BER always terminates.
  \end{theorem}

\end{itemize}

\end{slide}

% ------------------------------------------------------------------------
% Core ASN.1 and soundness property
%
\begin{slide}
\titre{Core \ASN and soundness property}

\centerline{\includegraphics[width=.85 \textwidth]{model-2.eps}}

\end{slide}

% ------------------------------------------------------------------------
% Getting formal
%
\begin{slide}
\titre{Getting formal}

We define the encoding with a \emph{system
of inference rules}. These are logical implications $P_1 \wedge P_2
\wedge \ldots \wedge P_n \Rightarrow C$ graphically represented as
\begin{mathpar}
\inferrule{P_1\\ P_2\\ \ldots\\ P_n}{C} 
\end{mathpar}
where the $P_i$ are the
\emph{premises} and $C$ is the \emph{conclusion}. When there is no
premise, $C$ is an \emph{axiom} and is simply noted $C$.

\myblue{An inference rule can be interpreted also from a computational
point of view}: in order to compute $C$, we need to compute the $P_i$
first (order is left unspecified).

\end{slide}

% ------------------------------------------------------------------------
% Inference rules (continued)
%
\begin{slide}
\titre{Inference rules (continued)}

The rules and axioms can contain unquantified variables
(\emph{free variables}). In this case they are implicitly universally
quantified ($\forall$) at the beginning. For instance
\[
\inferrule
  {P_1(x)\\ P_2(y)}
  {P(x,y)}\;\TirName{\textsc{Prop}}
\]
actually denotes the property \textsc{Prop} which is $\forall
x,y.P_1(x) \, \wedge \, P_2(y) \Rightarrow P(x,y)$. 

A system of inference rules is a non-ordered set of inference rules.

Given a system applying to one or more relations, these are defined as
the smallest relations satisfiying the rules.

\end{slide}

% ------------------------------------------------------------------------
% Smallest relations
% 
\begin{slide}
\titre{Smallest relations}

Let the following rules applying to the predicates
$\id{Even}(n)$ and $\id{Odd}(n)$:

\vspace*{-18pt}

\begin{mathpar}
\inferrule{}{\id{Even}(0)}
\and
\inferrule{\id{Odd}(n)}{\id{Even}(n+1)}
\and
\inferrule{\id{Even}(n)}{\id{Odd}(n+1)}
\end{mathpar}

\vspace*{-6pt}

By definition, we understand them as the propositions

\vspace*{-20pt}

\begin{gather*}
\id{Even}(0)\\
\forall n.\id{Odd}(n) \Rightarrow \id{Even}(n+1)\\
\forall n.\id{Even}(n) \Rightarrow \id{Odd}(n+1)\\
\end{gather*}

\vspace*{-18pt}

Many predicates satisfy these conditions, e.g. $\id{Even}(n)$ and
$\id{Odd}(n)$ true for all $n$. But the smallest satisfactory
predicates (the ones true the less often) are $\id{Even}(n) \triangleq
(n \mod{2} = 0)$ and $\id{Odd}(n) \triangleq (n \mod{2} = 1)$.

\end{slide}

% ------------------------------------------------------------------------
% Proof trees 
%
\begin{slide}
\titre{Proof trees}

A theorem is a judgement, i.e. a formal statement, a formula.

A proof of a theorem is a tree whose root (the conclusion) is the
theorem, the inner nodes are the conclusions of its subtrees and the
leaves are axioms. Such a tree is called a \emph{proof tree}. The
conclusion is drawn at the bottom of the page. For instance, here is
the proof of $\id{Odd}(3)$ with the previous system:

\begin{mathpar}
\inferrule
  {\inferrule
     {\inferrule
        {\id{Even}(0)}
        {\id{Odd}(1)}
     }
     {\id{Even}(2)}
  }
  {\id{Odd}(3)}
\end{mathpar}

\end{slide}

% ------------------------------------------------------------------------
% Back to our model
% 
\begin{slide}
\titre{Back to our model}

\begin{itemize}

  \item Let us note \myblue{$\coding{\Gamma}{v}{\T}{c}$} the judgement
  ``In the environment $\Gamma$, the value $v$ is encoded into the
  code $c$, following the type \T{}.''  (The environment models the
  \ASN module.)

  \item Let us note \myblue{$\decoding{\Gamma}{c}{\T}$} the decoding
  of $c$ in the environment $\Gamma$ according to type \T{}.

  \item We already noted \myblue{$v_1 \approx v_2$} the equivalence of
  values $v_1$ and $v_2$

  \item and \myblue{$c_1 \sim c_2$} the equivalence of codes $c_1$ and
  $c_2$.

\end{itemize}

The next step consists in defining these relations by means of systems
of inference rules.

\end{slide}

% ------------------------------------------------------------------------
% Formal model 
%
\begin{slide}
\titre{Formal model}

\vspace*{-5pt}

We can now rephrase formally our propositions:

\begin{itemize}

  \item Proposition ``All the BER encodings of a given value,
  according to a given type, are equivalent.'' becomes

  \begin{proposition}
    If $\coding{\Gamma}{v}{\T}{c_1}$ and $\coding{\Gamma}{v}{\T}{c_2}$
    then $c_1 \sim c_2$.
  \end{proposition}

  \item Proposition ``The decoding of two equivalent codes lead to two
  equivalent values.'' becomes

  \begin{proposition}[Equivalence entailment]
    $c_1 \sim c_2 \Longrightarrow \decoding{\Gamma}{c_1}{\T} \approx
    \decoding{\Gamma}{c_2}{\T}$
  \end{proposition}

  \item Theorem ``Let $v$ be a value of type \T. Then the BER decoding
  of \emph{any} BER encoding of $v$ is equivalent to $v$.'' becomes

  \begin{theorem}[Soundness]
    If $\coding{\Gamma}{v}{\T}{c}$ then $\decoding{\Gamma}{c}{\T}
    \approx v$.
  \end{theorem}

\end{itemize}

\end{slide}

% ------------------------------------------------------------------------
% Final steps
% 
\begin{slide}
\titre{Formal methods for the customer}

\begin{itemize}

  \item We prove the soundness theorem with a technique called
  \myblue{structural induction} (on the height of the proof trees) and
  \myblue{case analysis}.

  \item In case we are a standardization organisation or a tool seller
  we can say to the BER users: ``The formal design review concludes
  you can be pretty confident on the technology you use.''

  \item If we want a higher level of confidence, we can go further by
  refining our model, i.e. stating explicitly what was abstracted away
  and weakening assumptions.

\end{itemize}

\end{slide}

% ------------------------------------------------------------------------
% 
%
\begin{slide}
\titre{Formal methods for the software company}

\begin{itemize}

  \item In case we are a tool implementor, the \emph{reliable}
  back-end of an \ASN compiler can be derived from the previous
  work. Indeed our model, being based on inference rules, is actually
  close to an \myblue{operational model}, i.e. a real implementation,
  \myblue{based on a functional language}.

  \item In this case we do not need to publicise to our customer the
  usage of formal methods: these will ensure that the kernel of our
  software conforms to a formal specification, hence will have less
  bugs and will be easier to maintain, in the end.

  \item In industries other than the software industry, such formal
  design reviews are part of the normal process, e.g. the study of
  electromagnetic fields interference for the new Incheon subway line.

\end{itemize}

\end{slide}

%% \begin{slide}
%% \titre{Nouveaux axes de recherche}

%% Symbiose entre le \emph{web} et les \textsc{bd}: les sites \emph{web}
%% reposent de plus en plus sur des \textsc{bd}, les collections de pages
%% cha�n�es du \emph{web} sont une \textsc{bd} tentante.

%% Il faut r�inventer une approche \textsc{bd} du \emph{web} car celui-ci
%% contient des donn�es dites semi-structur�es, c-�-d. auto-descriptives
%% et irr�guli�res. Domaines affines:

%% \begin{itemize}

%%   \item syst�mes de types et analyse par contraintes des langages de
%%         requ�tes,

%%   \item syntaxes de transfert (codages pour une faible bande passante),

%%   \item interfonctionnement de syst�mes h�t�rog�nes (p. ex. ASN.1-XML),

%%   \item r�ing�nierie des sites \emph{web} (migration, XML
%%         \emph{wrappers}), 

%%   \item protocoles pour les nouveaux terminaux mobiles (assistants
%%         num�riques).

%% \end{itemize}

%% \end{slide}

\end{document}
